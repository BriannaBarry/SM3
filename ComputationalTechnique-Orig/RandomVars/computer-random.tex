 


One of the most surprising outcomes of the revolution in computing
technology has been the discovery of diverse uses for randomness 
in the analysis of data and in science generally.  Most young people
have little trouble with the idea of a computer generating random data;
they see it in computer games and simulations.  Older people,
raised with the idea that computers do mathematical operations and
that such operations are completely deterministic, sometimes find 
computer-generated randomness suspect.  Indeed, conventional algebraic
notation ($+$, $-$, $\sqrt{}$, $\cos$, and so on) has no
notation for ``generate at random.''

\index{C}{resampling}
\index{P}{resample@\texttt{resample}*}
\index{P}{Resampling!resample@\texttt{resample}*}
\index{C}{coin flip!simulation}
\index{P}{Simulations!coin flip}
One of the simplest operators for generating random events is
\code{resample}.  This takes two arguments: the first is a set of
items to choose from at random, the second is how many events
to generate.  Each item is equally likely to be
choosen.  For example, here is a simulation of a coin flip:
\begin{Schunk}
\begin{Sinput}
> coin = c("H","T")
> resample(coin, 5)
\end{Sinput}
\begin{Soutput}
[1] "H" "T" "T" "T" "H"
\end{Soutput}
\begin{Sinput}
> resample(coin, 5)
\end{Sinput}
\begin{Soutput}
[1] "H" "H" "H" "T" "T"
\end{Soutput}
\end{Schunk}
The first command creates an object holding the possible outcome of
each event, called \code{coin}.  The next command generated five
events, each event being a random choice of the outcomes in
\code{coin}.

Another example is rolling dice.  First, construct a set of the
possible outcomes: the numbers 1, 2, 3, 4, 5, 6.
\begin{Schunk}
\begin{Sinput}
> die = seq(1,6)
> die
\end{Sinput}
\begin{Soutput}
[1] 1 2 3 4 5 6
\end{Soutput}
\end{Schunk}
Then generate random events.  Here is a roll of two dice.
\begin{Schunk}
\begin{Sinput}
> resample(die,2)
\end{Sinput}
\begin{Soutput}
[1] 4 5
\end{Soutput}
\end{Schunk}

The \function{resample} function is also useful for selecting cases at
random from a data frame.  This use will be the basis for statistical
methods introduced in later chapters. 


\subsection{Random Draws from Probability Models}

Although \function{resample} is useful for random sampling, it can work
only with finite sets of possible outcomes such as H/T or 1/2/3/4/5/6
or the cases in a data frame.  By default in \function{resample}, the 
underlying probability model is
\newword{equiprobability} --- each possible outcome is equally likely.
You can specify another probability model by using the \code{prob=} argument
to \function{resample}. For instance, to flip coins that are very likely to come up heads:
\begin{Schunk}
\begin{Sinput}
> resample( coin, 10, prob=c(.9,.1))
\end{Sinput}
\begin{Soutput}
 [1] "H" "H" "H" "H" "H" "H" "H" "H" "H" "T"
\end{Soutput}
\end{Schunk}

R provides other operators that allow
draws to be made from outcome sets that are infinite.

\index{C}{probability model!sampling from}

For example, the \function{rnorm} function makes random draws from a
normal probability distribution.  The required argument tells how many
draws to make.  Optional, named arguments let you specify the mean and
standard deviation of the particular normal distribution that you
want.  To illustrate, here is a set of 15 random numbers from a normal
distribution with mean 1000 and standard deviation 75:
\begin{Schunk}
\begin{Sinput}
> samps = rnorm(15, mean=1000, sd=75)
> samps
\end{Sinput}
\begin{Soutput}
 [1]  977  941 1000  996  934 1029  995 1079  969  935 1017
[12]  992 1023 1022 1046
\end{Soutput}
\end{Schunk}
In this example, the output was assigned to an object \code{samps}
to facilitate some additional computations to the
values.  For instance, here is the mean and standard deviation of the
sample:
\begin{Schunk}
\begin{Sinput}
> mean(samps)
\end{Sinput}
\begin{Soutput}
mean 
 997 
\end{Soutput}
\begin{Sinput}
> sd(samps)
\end{Sinput}
\begin{Soutput}
  sd 
41.5 
\end{Soutput}
\end{Schunk}
Don't be surprised that the mean and standard deviation of the sample
don't match exactly the parameters that were set with the arguments
\code{mean=1000, sd=75}.  The sample was drawn at random and so the
sample statistics are going to vary from one sample to the next.  Part
of the statistical methodology to be studied in later chapters has to
do with determining how close the statistics calculated from a sample
are likely to be to the parameters of the underlying population.

Often you will generate very large samples.  In these situations you
usually don't want to display all the samples, just do calculations
with them.  The practical limits of ``large'' depend on the computer
you are using and how much time you are willing to spend on a
calculation.   For an operator like \code{rnorm} and the others to be
introduced in this chapter, it's feasible to generate samples of size
10,000 or 100,000 on an ordinary laptop computer.  
\begin{Schunk}
\begin{Sinput}
> samps = rnorm(100000, mean=1000, sd=75)
> mean( samps )
\end{Sinput}
\begin{Soutput}
mean 
1000 
\end{Soutput}
\begin{Sinput}
> sd( samps )
\end{Sinput}
\begin{Soutput}
  sd 
75.3 
\end{Soutput}
\end{Schunk}
Notice that the sample mean and standard deviation are quite close to
the population parameters in this large sample.  (Remember not to put commas in as punctuation in large numbers: it's \code{100000} not 100,000.)

The simulations that you will do in later chapters
will be much more elaborate than the simple draws here.  Even with 
today's computers, you will want to use only a few hundred trials.

\subsection{Standard Probability Models}

R provides a large set of operators like \code{rnorm} for different
probability models.  All of these operators work in the same way:
\begin{itemize}
\item Each has a required first argument that gives the number of
draws to make.

\item Each has an optional set of parameters that specify the
particular probability distribution you want.
\end{itemize}

All the operators start with the letter \code{r} --- standing for
``random'' --- followed by the name of the probability model:

\bigskip
\centerline{\begin{tabular}{llp{2.3in}l}
Family & R name & Parameters & \\\hline
Normal & \texttt{rnorm} & \texttt{mean},\texttt{sd} & continuous\\
Uniform & \texttt{runif} & \texttt{min},\texttt{max} & continuous\\
Binomial & \texttt{rbinom}& \texttt{size},\texttt{prob} & discrete\\
Poisson & \texttt{rpois}& Average rate (written \texttt{lambda})& discrete \\
Exponential & \texttt{rexp}& Same rate as in poisson but the parameter
is called \texttt{rate}.& continuous \\
Lognormal & \texttt{rlnorm}& Mean and sd of the natural
logarithm. \texttt{meanlog}, \texttt{sdlog}& continuous \\
$\chi^2$ & \texttt{rchisq} & Degrees of freedom (\texttt{df})&continuous \\
t & \texttt{rt} & Degrees of freedom (\texttt{df})&continuous\\
F & \texttt{rf} & Degrees of freedom in the numerator and in the
denominator (\texttt{df1}, \texttt{df2})&continuous\\
\end{tabular}}
\bigskip
\index{P}{rnorm@\texttt{rnorm}}
\index{P}{runif@\texttt{runif}}
\index{P}{rpois@\texttt{rpois}}
\index{P}{rexp@\texttt{rexp}}
\index{P}{rlnorm@\texttt{rlnorm}}
\index{P}{rchisq@\texttt{rchisq}}
\index{P}{rf@\texttt{rf}}
\index{P}{rt@\texttt{rt}}
\index{P}{rbinom@\texttt{rbinom}}
\index{P}{Probability!random generators}


To use these operators, you first must choose a particular probability
model based on the setting that applies in your situation.  This
setting will usually indicate what the population parameters should
be.  Some examples:
\begin{itemize}

\item You are in charge of a hiring committee 
that is  going to interview three candidates selected from a
population of job applicants that is 63\% female.  How many of the
interviewees will be female?  Modeling this as random selection from
the applicant pool, a binomial model is appropriate.  The \code{size}
of each trial is 3, the probability of being female is 63\%:
\begin{Schunk}
\begin{Sinput}
> samps = rbinom(40, size=3, prob=0.63)
> samps
\end{Sinput}
\begin{Soutput}
 [1] 1 3 2 2 1 2 3 1 1 1 1 2 2 2 2 1 2 1 2 2 2 1 3 1 2 1 2 1
[29] 2 2 1 3 1 3 2 2 2 2 2 2
\end{Soutput}
\end{Schunk}
There are 40 trials here, since the first argument was set to 40.
Remember, each of the trials is a simulation of one hiring event.  In
the first simulated event, two of the interviewees were female; in the
third only one was female.  Typically, you will be summarizing all the
simulations, for example to see how likely each possible outcome is.
\begin{Schunk}
\begin{Sinput}
> table(samps)
\end{Sinput}
\begin{Soutput}
samps
 1  2  3 
14 21  5 
\end{Soutput}
\end{Schunk}

\item You want to simulate the number of customers who come into a
store over the course of an hour.  The average rate is 15 per hour.
To simulate a situation where customers arrive randomly, the poisson
model is appropriate:
\begin{Schunk}
\begin{Sinput}
> rpois(25, lambda=15)
\end{Sinput}
\begin{Soutput}
 [1] 18 15 13 15 14 15 19 10 17 25 13 13  8 19 14 13  8 19
[19] 13 23 12 16 13 11 19
\end{Soutput}
\end{Schunk}

\item You want to generate a simulation of the interval between
earthquakes as in Example \ref{example:earthquake-intervals}.  To
simulate the random intervals with a typical rate of 0.03 earthquakes
per year, you would use
\begin{Schunk}
\begin{Sinput}
> rexp( 15, rate=0.03 )
\end{Sinput}
\begin{Soutput}
 [1]  16.48  13.43  18.74  44.05 129.50  66.52  39.94   5.31
 [9]  14.65  21.42  72.14  45.95  21.64  14.64  26.51
\end{Soutput}
\end{Schunk}

Notice the huge variation in the intervals, from less than half a year
to almost 90 years between earthquakes.

\end{itemize}

\subsection{Coverage Intervals}

\index{C}{coverage interval!computing}

You will often need to compute coverage intervals in order to describe 
the range of likely outcomes from a random process.  R provides a
series of operators for this purpose; a separate operator for each
named probability model.  The operators all begin with \code{q},
standing for \newword{quantiles}.  In all cases, the first argument is
the set of quantiles you want to calculate for the particular
probability model.  The optional named arguments are the parameters.

\index{C}{coverage interval!level}
Remember
that to find a 95\% coverage interval you need the 0.025 and 0.975
quantiles.  For a 99\% interval, you need the 0.005 and 0.995
quantiles.

To illustrate, here are 95\% coverage intervals for a few
probability models. 
\begin{itemize}
\item A normal distribution with mean 0 and standard deviation 1:
\begin{Schunk}
\begin{Sinput}
> qnorm( c(0.025, 0.975), mean=0, sd=1)
\end{Sinput}
\begin{Soutput}
[1] -1.96  1.96
\end{Soutput}
\end{Schunk}
> 

\item The hiring committee situation modelled by a binomial
distribution with \code{size=3} and \code{prob=0.63}:
\begin{Schunk}
\begin{Sinput}
> qbinom( c(0.025, 0.975), size=3, prob=0.63)
\end{Sinput}
\begin{Soutput}
[1] 0 3
\end{Soutput}
\end{Schunk}

Perhaps you are surprised to see that the coverage interval includes
all the possible outcomes.  That's because the number of cases in each
trial ($n=3$) is quite small. 

\item The number of customers entering a store during an hour 
as modelled by a poisson
distribution with an average rate of 15 per hour.
\begin{Schunk}
\begin{Sinput}
> qpois( c(0.025, 0.975), lambda=15)
\end{Sinput}
\begin{Soutput}
[1]  8 23
\end{Soutput}
\end{Schunk}

\item The interval between earthquakes modelled by an exponential
distribution with a typical rate of 0.03 earthquakes per year:
\begin{Schunk}
\begin{Sinput}
> qexp( c(.025, .975), rate=0.03)
\end{Sinput}
\begin{Soutput}
[1]   0.844 122.963
\end{Soutput}
\end{Schunk}
\end{itemize}

\index{P}{quantiles!q-functions}
\index{P}{Probability!q operators}
\index{C}{quantiles!computing}

You can also use the \code{q} operators to find the value that would
be at a particular percentile.  For example, the exponential model with \code{rate=0.03} 
gives the 25th percentile of
the interval between earthquakes as:
\begin{Schunk}
\begin{Sinput}
> qexp( .25, rate=0.03)
\end{Sinput}
\begin{Soutput}
[1] 9.59
\end{Soutput}
\end{Schunk}
A quarter of the time, the interval between earthquakes will be 9.59
years or less.

It's entirely feasible to calculate percentiles and 
coverage intervals by combining
the random-number generators with \code{quantile}.  For example, here
is the 95\% coverage interval from a normal distribution with mean 0
and standard deviation 1:
\begin{Schunk}
\begin{Sinput}
> samps = rnorm(10000, mean=0, sd=1)
> qdata( c(.025, .975), samps )
\end{Sinput}
\begin{Soutput}
 2.5% 97.5% 
-1.99  1.95 
\end{Soutput}
\end{Schunk}
The disadvantage of this approach is that it is a simulation and the
results will vary randomly.  By making the sample size large enough
--- here it is $n=10000$ --- you can reduce the random variation. 
Using the \code{q} operators uses mathematical analysis to give 
you what is effectively an
infinite sample size.  For this reason, it's advisable to use the
\code{q} operators when you can.  However, for many of the techniques to
be introduced in later chapters you will have to generate a random
sample and then apply \code{quantile} to approximate the coverage intervals.

\subsection{Percentiles}

\index{P}{percentile!p-functions}
\index{C}{percentile!computing}
\index{P}{Probability!p operators}
\index{P}{Probability!q operators}

A percentile computation applies to situations where you have a
measured value and you want to know where that value ranks relative to
the entire set of possible outcomes.  You have already seen
percentiles computed from samples; they also apply to probability models.

It's easy to confuse percentiles with quantiles because they are so
closely related.
Mathematically, the percentile operators are the inverse of the
quantile operators. 
To help you remember which is which, it's helpful to distinguish them
based on the type of argument that you give to the operator:
\begin{description}
\item[Percentile] The input 
argument is a measured value, something that could be the
output of a single draw from the probability distribution. The output
is always a number between 0 and 1 --- a percentile. 

\item[Quantile] The input is a percentile, a number between 0 and 1.
The output is on the scale of the measured variable.
\end{description}


Example:  You have just gotten your score, 670, on a professional school
admissions test.  According to the information published by the
testing company, the scores are normally distributed with a mean of
600 and a standard deviation of 100.  So, your ranking on the test, as
indicated by a percentile, is:
\begin{Schunk}
\begin{Sinput}
> pnorm(670, mean=600, sd=100)
\end{Sinput}
\begin{Soutput}
[1] 0.758
\end{Soutput}
\end{Schunk}
Your score is at about the 75th percentile.  

Example: Unfortunately, the professional school that you want to go to
accepts only students with scores in the top 15 percent.  Your score,
at 75.8\%, isn't good enough.  So, you will study some more and take
practice tests until your score is good enough.
How well will you need to score to reach the 85th percentile?
\begin{Schunk}
\begin{Sinput}
> qnorm(0.85, mean=600, sd=100)
\end{Sinput}
\begin{Soutput}
[1] 704
\end{Soutput}
\end{Schunk}


