
\subsubsection{Reading CSV Files into R}
\label{sec:read-csv}

For the data sets associated with this book and its exercises, an easy way 
to import data into R is with the \code{ISMdata} 
operator that's
included in the extensions to R found in the \code{ISM.Rdata} workspace file.
(See Section \ref{sec:R-customizations} on page
\pageref{sec:R-customizations}.  You must load the workspace file before you
can use \code{ISMdata}.)  

\index{P}{ISMdata}
\index{P}{Data!ISMdata@\texttt{ISMdata}*}
\index{C}{Data files!reading with ISMdata}
\index{C}{Data files!on Internet}
\index{C}{Internet!data files on}

\code{ISMdata} lets you refer to a file by a short name in quotes without worrying about where it is located.
It knows where to locate the data files used with this
book, whether they be on your computer or on the web.  For example, the data set \code{hdd-minneapolis.csv} is stored on the Internet.  You can read it into an objected named \code{hdd} with this statement:
\begin{verbatim}
> hdd = ISMdata("hdd-minneapolis.csv")
Not in library.  Trying to find it on the web ...
File was read from the web.
\end{verbatim}
If you do not have an Internet connection, then \code{ISMdata} will be able to locate only 
files on your computer.

It is possible to use \code{ISMdata} to read in your own data that you have created and stored in CSV files.  To do this, you need to tell \code{ISMdata} where the data is located.
The easiest way to do this is to use a mouse-based file navigator.  To do this, invoke \code{ISMdata} with no input argument:
\begin{verbatim}
> swim = ISMdata()
\end{verbatim}
Since there was no character string file name given as an argument, 
\code{ISMdata} brings up a file navigator to let you select the file
interactively:

\bigskip
\centerline{\graphicsfile[width=2.5in]{select-file.png}}

\centerline{\parbox{2.5in}{\sf Selecting a file interactively.}}
\bigskip

Selecting and opening the desired file will cause it to be read into R
as a data frame.  Make sure to choose the correct CSV file.  If you
choose some other sort of file by accident, \code{ISMdata} will
struggle to read it in, displaying various junk on your screen.

The statement above will cause the resulting data frame to be called \code{swim}.  But be careful. If you accidentally choose a different file (e.g., \code{kidsfeet.csv}) the data from that file will be read in and stored under the name \code{swim}, even though they have nothing to do with swimming.


