The first step in analyzing data in R is to \newword{read in the data}
from a file.  For data stored as a CSV file, 
this is a simple, automatic process.  All you need to do
is tell R where the data file is located.  There are two principal
sorts of locations:
\begin{description}

\item[On your computer] This is a good choice for data that you create
  yourself, or if you always use your own computer and you need to do
  your work when your computer is not connected to the Internet.

  It's a good idea to create a directory to hold the CSV data files
  that you will need to use.  Then copy the data files you will need
  onto your computer either by downloading them at some time when the
  computer is connected to the Internet, or by using some portable
  memory medium such as a flash drive or a CD/DVD disk.  Information
  on the location of the data files is given at the web site
  \url{www.macalester.edu/~kaplan/ISM}


\item[On the Internet] This is a good choice if you use shared
  computers (such as laboratory or classroom computers).  The data
  files used in this book will be identified with a URL link that will
  work in any browser. 
\end{description}

Whether a data file is located on your computer or on the Internet, the
process of reading it into R is the same.  For CSV files, the standard 
operator to use is \code{read.csv}.  This takes as an argument the name and
location of the file.  The output of \code{read.csv} is a data frame
--- you will want to store this in an object. 

The basic use of \code{read.csv} is simple.  For instance:
\begin{verbatim}
> swim = read.csv("swim100m.csv")
\end{verbatim}
This will read in the contents of the file \code{swim100m.csv}, and
create a data frame named \code{swim}.
 
Whether the above statement will work depends on where the file is
located.  Most likely, it will generate an error message unless you
have put the file in a place that R knows about.
\begin{verbatim}
> swim = read.csv("swim100m.csv")
Error 
  cannot open file 'swim100m.csv': 
  No such file or directory
\end{verbatim}  
Unfortunately, much of the content of error messages is hard to interpret.  
It's good to get in the habit of scanning such messages, when they appear, to 
look for some part of it that makes sense.  Here, ``no such file or
directory'' is a good hint that R can't find your data file.
 
When you create a data file or download one to your computer, you put
it in a ``directory'' (sometimes called a ``folder'').  For example,
files that you download using a web browser are often placed into a
directory called the {\em desktop} or {\em downloads}.  To help
organize their files, people often create new directories for each
project, storing the files related to that project in the new
directory.

However you choose to organize things, you need to tell R where to
look for the files you create or download.  On the Windows version of
R, you can do this by using the {\sc FILE/Change Working Directory}
menu in R.  On Macintosh, this is done using the {\sc MISC/Change
  Working Directory} menu item.  Either way, the result is to bring up
a standard file-system navigator window that lets you select the
appropriate directory.

