
\subsection{Reading Tabular Data into R}

Data used in statistical modeling are usually organized into tables, 
often created using spreadsheet software. Most people
presume that the same software used to create a table of data should
be used to display and analyze it.  This is part of the reason for the
popularity of spreadsheet programs such as Excel and Google Spreadsheets.

\index{C}{Excel}
\index{C}{Google Spreadsheets}
\index{C}{spreadsheet software}
\index{P}{Importing data to R}
\index{P}{Data!importing to R}
\index{P}{ISMdata}
\index{P}{CSV files!reading}
\index{P}{Data!ISMdata@\texttt{ISMdata}*}
\index{P}{read.csv@\texttt{read.csv}}
\index{P}{Data!read.csv@\texttt{read.csv}}
\index{P}{reading CSV files}

For serious statistical work, it's helpful to take another approach that
strictly separates the processes of data collection and of data
analysis: use one program to create data files and another program to
analyze the data stored in those files.  
By doing this, one guarantees that the original data are
not modified accidentally in the process of analyzing them.  This also
makes it possible to perform many different analyses of the data;
modelers often create and compare many different models of the same
data.

\subsubsection{Reading Tabular Data into R}

Data is central to statistics, and the tabular arrangement
of data is very common.  Accordingly, R provides a large number of
ways to read in tabular
data.  These vary depending on how the data are stored, where they are
located, etc.  To help keep things as simple as possible, the \pkg{mosaic}
package in R provides an operator, \function{fetchData} that allows
you to access data files in tabular format on your computer as well as data stored on
repositories such as the one associated with this book or one that a
course instructor might set up for his or her students.

\index{P}{fetchData}

\label{sec:loading-mosaic}
To use \function{fetchData}, as well as a variety of other operators
that you will encounter later, you need to load the \pkg{mosaic}
package. 
\begin{Schunk}
\begin{Sinput}
> require(mosaic)
\end{Sinput}
\end{Schunk}
You need do this only once in each session of R, and on systems such
as RStudio the package will generally be reloaded automatically.
(If you get an error message, it's likely that the \pkg{mosaic} package
has not been installed on your system.  Use the package installation
menu in R to install \pkg{mosaic}, after which the \function{require}
function will load the package.)

Reading in a data table that's been connected with \function{fetchData}
is simply a matter of knowing the name of the data set.  For instance,
one data table used in examples in this book is
\texttt{swim100m.csv}.  To read in this data table and create an
object in R that contains the data, use a command like this:
\datasetSwimming
\begin{Schunk}
\begin{Sinput}
> swim = fetchData("swim100m.csv")
\end{Sinput}
\end{Schunk}

The \texttt{csv} part of the name in \texttt{"swim100m.csv"} indicates
that the file has been stored in a particular data format,
\newword{comma-separated values} that is handled by spreadsheet
software as well as many other kinds of software.

The part of this command that requires creativity is choosing a
name for the R object that will hold the data.  In the above command
it is called \texttt{swim}, but you might prefer another name, e.g.,
\texttt{s} or \texttt{sdata} or even \texttt{ralph}.  Of course, it's
sensible to choose names that are short, easy to type and remember,
and remind you what the contents of the object are about.

To help you identify data tables that can be accessed through
\function{fetchData}, examples in this book will be marked with a flag
containing the name of the file containing the data file.  The files
themselves are either built in to the \pkg{mosaic} package or
available automatically through the web site for this book.

\subsubsection{Data Frames}
\index{P}{data@\texttt{data}}
\index{C}{data frame|(}

The type of R object created by \function {fetchData} is called a
\newword{data frame} and is essentially a tabular layout. To illustrate , here are the first several cases
of the \texttt{swim} data frame created by the previous use of \function{fetchData()}:
\begin{Schunk}
\begin{Sinput}
> head(swim)
\end{Sinput}
\begin{Soutput}
  year time sex
1 1905 65.8   M
2 1908 65.6   M
3 1910 62.8   M
4 1912 61.6   M
5 1918 61.4   M
6 1920 60.4   M
\end{Soutput}
\end{Schunk}
Note that the \function{head} function, one of several functions that
operate on data frames, takes the R object that you
created, not the quoted name of the data file.  

Data frames, like tabular data generally, involve variables and
cases.  In R, each of the variables is given a name.  You can refer to
the variable by name in a couple of different ways.  To see the
variable names in a data frame, something you might want to do to
remind yourself of how names a spelled and capitalized, use the
\function{names} function:
\begin{Schunk}
\begin{Sinput}
> names(swim)
\end{Sinput}
\begin{Soutput}
[1] "year" "time" "sex" 
\end{Soutput}
\end{Schunk}
Another way to get quick information about the variables in a data
frame is with \function{summary}:
\begin{Schunk}
\begin{Sinput}
> summary(swim)
\end{Sinput}
\begin{Soutput}
      year           time      sex   
 Min.   :1905   Min.   :47.8   F:31  
 1st Qu.:1924   1st Qu.:53.6   M:31  
 Median :1956   Median :56.9         
 Mean   :1952   Mean   :59.9         
 3rd Qu.:1976   3rd Qu.:65.2         
 Max.   :2004   Max.   :95.0         
\end{Soutput}
\end{Schunk}

To see how many cases there are in a data frame, use \function{nrow}:
\begin{Schunk}
\begin{Sinput}
> nrow(swim)
\end{Sinput}
\begin{Soutput}
[1] 62
\end{Soutput}
\end{Schunk}

\subsubsection{Variables in Data Frames}

\index{C}{data frame!variables of}
\index{C}{columns!in data frames}
\index{P}{columns!in data frame}
\index{C}{variables!in data frames}
\index{P}{variables!in data frame}
\index{P}{.@{\texttt{\$}}}
\index{P}{Data!access variables with \$}
Perhaps the most common operation on a data frame is to refer to the
values in a single variable.  The two ways you will most commonly use
involve functions with a \texttt{data=} argument and the direct use of
the \verb+$+ notation.

Most of the statistical modeling functions you will encounter in this
book are designed to work with data frames and allow you to refer
directly to variables within a data frame.  For instance:
\begin{Schunk}
\begin{Sinput}
> mean(year, data = swim)
\end{Sinput}
\begin{Soutput}
[1] 1952
\end{Soutput}
\begin{Sinput}
> min(year, data = swim)
\end{Sinput}
\begin{Soutput}
[1] 1905
\end{Soutput}
\end{Schunk}

The \texttt{data=} argument tells the function which data frame to pull the
variable from.  Leaving off that argument leads to an error:
\begin{Schunk}
\begin{Sinput}
> mean(year)
\end{Sinput}
\end{Schunk}
\begin{Schunk}
\begin{Soutput}
Error in mean(year) : 
   error in evaluating the argument 'x' 
\end{Soutput}
\end{Schunk}
% The following was replaced with the Schunk for formatting purposes.

Of course, you know that the variable \texttt{year}
is defined within the data frame \texttt{swim}, but you have to give
this information to the function \function{mean} explicitly, otherwise
it doesn't know where to find the variable.  

You can also refer to the variable using the \verb+$+ notation, e.g.
\begin{Schunk}
\begin{Sinput}
> mean(swim$year)
\end{Sinput}
\begin{Soutput}
mean 
1952 
\end{Soutput}
\end{Schunk}
Think of this as referring to the variable by both its family name
(the data frame's name,``swim'') and its given name (``year''), something
like Einstein\$Albert.

The advantage of the \texttt{data=} approach becomes evident when you
construct statements that involve more than one variable within a data
frame.  For instance, here's a calculation of the mean year separately
for the different sexes:
\begin{Schunk}
\begin{Sinput}
> mean(year ~ sex, data = swim)
\end{Sinput}
\begin{Soutput}
  sex    S  N Missing
1   F 1951 31       0
2   M 1954 31       0
\end{Soutput}
\end{Schunk}
You will see much more of the \verb+~+ (``tilde'') 
starting in Chapter \ref{chap:simple-models}.  It's the R notation for ``broken down by'' or ``versus.''

Both the \function{mean} and \function{min} functions know to look in
the data frame when interpreting variables, but not all R functions
--- particularly older ones --- are designed this way.\footnote{The
  ability of \function{mean}, \function{median}, and similar
  functions to handle the \texttt{data=} format is provided by the
  \pkg{mosaic} package.}  When you encounter a function that can't
handle the \texttt{data=} format, use the \texttt{\$} notation. 
For example, this works
\begin{Schunk}
\begin{Sinput}
> mean(sqrt(year), data = swim)
\end{Sinput}
\begin{Soutput}
[1] 44.2
\end{Soutput}
\end{Schunk}
But not this:
\begin{Schunk}
\begin{Sinput}
> sqrt(year, data = swim)
\end{Sinput}
\end{Schunk}
\begin{Schunk}
\begin{Soutput}
Error in try(sqrt(year, data = swim)) : object 'year' not found
\end{Soutput}
\end{Schunk}



\index{P}{length@\texttt{min}}
\index{P}{mean@\texttt{mean}}
\index{P}{max@\texttt{max}}

\subsubsection{Adding a New Variable}

\index{C}{data frame!adding a new variable}
\index{C}{variable!adding to a data frame}

Sometimes you will compute a new quantity from the existing variables and 
want to treat this as a new variable.  Adding a new variable to a data
frame can be done with the \texttt{\$} notation. 
For instance, here is how to create a new variable in \texttt{swim}
that holds the \texttt{time} converted from minutes to units of seconds:
\begin{Schunk}
\begin{Sinput}
> swim$minutes = swim$time/60
\end{Sinput}
\end{Schunk}
The new variable appears just like the old
ones:
\begin{Schunk}
\begin{Sinput}
> names(swim)
\end{Sinput}
\begin{Soutput}
[1] "year"    "time"    "sex"     "minutes"
\end{Soutput}
\end{Schunk}
You could also, if you want, redefine an existing variable, for
instance:
\begin{Schunk}
\begin{Sinput}
> swim$time = swim$time/60
\end{Sinput}
\end{Schunk}
Such assignment operations do not change the original file from which
the data were read, only the data frame in the current session of R.
This is an advantage, since it means that your data in the data file
stay in their original state and therefore won't be corrupted by
operations made during analysis.

\subsection{Sampling from a Sample Frame}

\index{C}{sampling!with resample}
\index{C}{sampling!with shuffle}

Much of statistical analysis is concerned with the consequences of
drawing a sample from the population.  
Ideally, you will
have a sampling frame that lists every member of the population from
which the sample is to be drawn.  With this in hand, you could treat
the individual cases in the sampling frame as if they were cards in a
deck of hands.  To pick your random sample, shuffle the deck and deal
out the desired number of cards.  

When doing real work in the field, you would use the randomly dealt
cards to locate the real-world cases they correspond to.  Sometimes in
this book, however, in order to let you explore the consequences of
sampling, you will select a sample from an existing data set.  The
\function{deal} function performs this, taking as an argument the data
frame to be used in the selection and the number of cases to sample.

For example, the \texttt{kidsfeet.csv} data set \datasetFeet has
$n=39$ cases.
\begin{Schunk}
\begin{Sinput}
> kids = fetchData("kidsfeet.csv")
> nrow(kids)
\end{Sinput}
\begin{Soutput}
[1] 39
\end{Soutput}
\end{Schunk}
Here's how to take a random sample of five of the cases:
\begin{Schunk}
\begin{Sinput}
> deal(kids, 5)
\end{Sinput}
\begin{Soutput}
       name birthmonth birthyear length width sex biggerfoot domhand
8   Caitlin          6        88   23.0   8.8   G          L       R
17 Caroline         12        87   24.0   8.7   G          R       L
29     Mike         11        88   24.2   8.9   B          L       R
6    Scotty          3        88   25.7   9.7   B          R       R
20  Heather          3        88   25.5   9.5   G          R       R
\end{Soutput}
\end{Schunk}


\index{C}{sampling}
\index{C}{resampling!with replacement}
\index{P}{resample@\texttt{resample}*}
\index{P}{deal@\texttt{deal}*}
\index{P}{Resampling!resample@\texttt{resample}*}

The results returned by \function{deal} will never contain the same
case more than once, just as if you were dealing cards from a shuffled
deck.  In contrast, \code{resample} replaces each case after it is
dealt so that it can appear more than once in the result.  You
wouldn't want to do this to select from a sampling frame, but it turns
out that there are valuable statistical uses for this 
sort of \newworddef{sampling with replacement}{sampling!with
  replacement}.  You'll make use of re-sampling in Chapter \ref{chap:statistical-inference}.


\begin{Schunk}
\begin{Sinput}
> resample(kids, 5)
\end{Sinput}
\begin{Soutput}
         name birthmonth birthyear length width sex biggerfoot domhand
15      Julie         11        87   26.0   9.3   G          L       R
31   Danielle          6        88   24.0   9.3   G          L       R
29       Mike         11        88   24.2   8.9   B          L       R
38     Hayley          1        88   21.6   7.9   G          R       R
29.1     Mike         11        88   24.2   8.9   B          L       R
\end{Soutput}
\end{Schunk}

Notice that Mike was sampled twice.

