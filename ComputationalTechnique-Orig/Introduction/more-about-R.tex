This appendix will contain more details on R, things that the reader doesn't need to read on the first pass through.


\label{chap:more-about-r}

\subsection{Setting R Up}

REVISE THIS

\label{sec:setting-r-up}
You can skip this section if you want to jump ahead to read about how R works. 

\index{C}{R software!installation}

Better, though, if you try out the commands as they are shown, using
the R software on your own computer. 

To do this, you will need to start the R software.  If R is not
already installed on your computer (this is likely if you are using
your own computer and have never used R before), the first step is to
install it. 

If you are used to installing software, you will find R follows the
usual pattern.  
\begin{enumerate}
\item Use a web browser to go to \url{www.r-project.org}. Select
the Download/CRAN link on the left of the page. This will bring you to
a list of download sites.

\item  Choose one in your own region of the
world.  This will bring up a page with a choice of Linux, Mac OS X, or
Windows.  Choose whichever is appropriate for your computer.

\item  For Windows: choose the link for the ``base'' distribution, and
  then download the link that looks like ``R-2.9.0-win32.exe.''  The
  name may be slighly different, depending on what new versions have
  been released.  Run this program, accepting the defaults.

  For Macintosh, follow the link that looks like ``R-2.9.0.dmg'' and
  follow the instructions.  Again, the name may be slightly different,
  depending on what new versions have been released.

  If you are using Linux, you probably don't need any instructions on
  how to install software.
\end{enumerate}  

\subsubsection{Defining your own operators}

Occasionally, you may need to define your own operators.  This is
convenient if you need to repeat an operation many times or if you
need to define a mathematical function.
\index{P}{function@\texttt{function}}
\index{P}{R Syntax!function@\texttt{function}}

It's important to keep in mind the difference between an operator and
a command.  A command is an instruction to perform a particular
computation on a particular input argument or set of input arguments.
The input arguments always have to be values, though of course you can
refer to the value by giving the name of an object that has already
been assigned a value.

In contrast, in defining an operator, you can treat the arguments
abstractly; just a name without a value having been assigned.  To
illustrate, here is a command that creates the mathematical
function $f(x) = 3 x^2 + 2$ and stores it in an operator named \code{f}:
\begin{Schunk}
\begin{Sinput}
> f = function(x) {
     3 * x^2 + 2
 }
\end{Sinput}
\end{Schunk}
One you have defined the function, you can invoke it in the standard
way.  For example:
\begin{Schunk}
\begin{Sinput}
> f(3)
\end{Sinput}
\begin{Soutput}
[1] 29
\end{Soutput}
\begin{Sinput}
> f(10)
\end{Sinput}
\begin{Soutput}
[1] 302
\end{Soutput}
\end{Schunk}

\index{C}{syntax!function definition}
There are some novel features to the syntax used to define a new
operator.  First, the arguments to \code{function} aren't treated as
values but as pure names.  Second, the contents of the 
curly braces \verb+{+ and \verb+}+ ---  the
function contents --- are the commands that will be 
evaluated when the function is invoked.

It doesn't matter what names you use in the function contents so long
as they match the names used in the arguments to \code{function}.  For
example, here is another operator, called \code{g}, that will perform
exactly the same computation as \code{f} when invoked:
\begin{Schunk}
\begin{Sinput}
> g = function(marge) {
     3 * marge^2 + 2
 }
\end{Sinput}
\end{Schunk}

\index{C}{invoking an operator}
When you invoke an operator, the interpreter carries out several
steps.  Consider the invocation
\begin{Schunk}
\begin{Sinput}
> g(7)
\end{Sinput}
\begin{Soutput}
[1] 149
\end{Soutput}
\end{Schunk}
In carrying out this command, the interpreter will:
\begin{enumerate}
\item Temporarily define or redefine an object \code{marge} that has
  the value 7.
\item Execute the function contents.
\item Return the value of these contents as the return value of the
  command.
\item Discard the definition or redefinition in (1).
\end{enumerate}

Operators can have more than one argument.  For instance, here is an
operator \code{hypotenuse} that computes the length of the hypotenuse
of a right triangle given the lengths of the legs
\begin{Schunk}
\begin{Sinput}
> hypotenuse = function(a, b) {
     sqrt(a^2 + b^2)
 }
\end{Sinput}
\end{Schunk}

When programmers create new operators that they expect to use on many
different occasions, they put the commands
to define the operators into a text file called a 
\newword{source file}.  
This file can be read into R using a special operator, called
\code{source} that causes the commands to be executed, thereby
defining the new operators.
\index{P}{source@\texttt{source}}
\index{P}{R System!source@\texttt{source}}

\subsubsection{Saving and Documenting Your Work}

\index{P}{Saving your work in R}
\index{P}{R System!saving your work}

Up to now, the commands used as examples have been simple one-line
statements.  As you work further, you will build more elaborate
computations by combining simpler ones.  It will become important to
be able to document what you have done, providing a record so that
others can confirm your results and so that you and others can modify
your work as needed.

One way in which a record is created of your interaction with the
computer is the dialog in the interpreter console itself.  In some
ways this is analogous to a document created by a word processor: for
example, you can copy the contents and paste it into another document.

But the idea of dialog-as-document is flawed.  For example, in a
word-processor, when you correct a mistake the old version is erased.
But in the R dialog, to fix a mistake you give a new command --- the
old, mistaken command is still there in the dialog.


You should keep in mind that there are several different components,
some of which are more appropriate than others for your documentation.

\begin{description}
\item[Your Commands]  The commands that you execute are what defines
  the computation being performed.  These commands themselves are a
  valuable form of documentation. 

\item[The objects you create] These objects, and the values that are
  stored in them, reflect the \newword{state} of the computation.  If
  you want to pick up on your work where you left off, you can save
  these objects.  This is called ``saving the \newword{workspace}.''

\item[Side effects]  This refers to the output printed by the
  interpreter and plots.  Sometimes you will want to include this in your
  documentation, but usually just select elements.\index{C}{side effects}
\end{description}

\subsection{Using Customizations to R}

\label{sec:R-customizations}

\index{C}{ISM.Rdata}
\index{P}{ISM.Rdata}
\index{P}{R System!ISM.Rdata workspace}

One of the features that makes R so powerful is that new commands can easily be
added to the system.  This makes it possible, for instance, to
customize the software to make routine tasks easier.  Such
customizations have been written specifically for the people following
this book.  Most of them are contained in the \texttt{mosaic}
package. 

To use the commands in \texttt{mosaic}, you need to instruct R to
``load'' the package.  This can be done with the following command:
\begin{Schunk}
\begin{Sinput}
> require(mosaic)
\end{Sinput}
\end{Schunk}



If you get an error message, it is likely because the package has not
yet been installed on your system.  Installation is a one-time process
described in Appendix \ref{appendix:installing-R}.


\subsubsection{Logical Data and Logical Operators}

\index{C}{logical!operators|(}
\index{C}{data!selecting subsets}
Many computations involve selections of subsets of data that meet some
criterion.  For example, in studying the health of newborn babies, you
might want to focus only on those below a certain birthweight or
perhaps those babies whose mother smoked during pregnancy.  The
question of whether the case satisfies the criterion boils down to a
yes-or-no answer.

Logic is the study of valid inference; it is intimately tied up with
the idea of truth versus falsehood.  In computer languages,
\newworddef{logical data}{logical!data type} refers to a type of data that can represent
whether something is true or false.  To illustrate, consider the
simple sequence 1 to 7
\begin{Schunk}
\begin{Sinput}
> x = seq(1, 7)
\end{Sinput}
\end{Schunk}
Now a simple question about the values in \code{x}: Are any of them
greater than $\pi$?  Here's how you can ask that question:
\begin{Schunk}
\begin{Sinput}
> x > pi
\end{Sinput}
\begin{Soutput}
[1] FALSE FALSE FALSE  TRUE  TRUE  TRUE  TRUE
\end{Soutput}
\end{Schunk}

A computer command like \code{x > pi} is not the same as an
algebraic statement like $x > \pi$.  The algebraic statement describes
the relationship between $x$ and $\pi$, namely that $x$ is greater than
$\pi$.  The computer command asks a question: Is the value of
\code{x} greater than the value of \code{pi}?  Asking this question
invokes a computation; the returned value is the answer to the
question, either \code{TRUE} or \code{FALSE}.  

Here are some of the operators for asking such questions:

\index{C}{comparison operators}
\index{C}{logical!comparison operator}
\index{P}{.@\texttt{==}}
\index{C}{character string!comparison}
\index{P}{R Syntax!.@\texttt{==}}

\bigskip
\centerline{\begin{tabular}{ll}
\code{x < y} & Is \code{x} less than \code{y}?\\
\code{x <= y} & Is \code{x} less than or equal to \code{y}?\\
\code{x > y} & Is \code{x} greater than \code{y}?\\
\code{x >= y} & Is \code{x} greater than  or equal to \code{y}?\\
\code{x == y} & Is \code{x} equal to \code{y}?\\
\code{x != y} & Is \code{x} unequal to \code{y}?\\
\end{tabular}}
\bigskip
Notice the double equal signs in \code{x == y}.  A single equal sign
would be the assignment operator.

\index{P}{comparison!operators}

Mostly, these comparison operators apply to numbers.  The \code{==}
and \code{!=} operators also apply to character strings.  To
illustrate, I'll define two objects \code{v} and \code{w}:
\begin{Schunk}
\begin{Sinput}
> v = "hello"
> w = seq(1, 15, by = 3)
> w
\end{Sinput}
\begin{Soutput}
[1]  1  4  7 10 13
\end{Soutput}
\begin{Sinput}
> w < 12
\end{Sinput}
\begin{Soutput}
[1]  TRUE  TRUE  TRUE  TRUE FALSE
\end{Soutput}
\begin{Sinput}
> w > 7
\end{Sinput}
\begin{Soutput}
[1] FALSE FALSE FALSE  TRUE  TRUE
\end{Soutput}
\begin{Sinput}
> w != 4
\end{Sinput}
\begin{Soutput}
[1]  TRUE FALSE  TRUE  TRUE  TRUE
\end{Soutput}
\begin{Sinput}
> v == "goodbye"
\end{Sinput}
\begin{Soutput}
[1] FALSE
\end{Soutput}
\begin{Sinput}
> v != "hi"
\end{Sinput}
\begin{Soutput}
[1] TRUE
\end{Soutput}
\begin{Sinput}
> v == "hello"
\end{Sinput}
\begin{Soutput}
[1] TRUE
\end{Soutput}
\begin{Sinput}
> v == "Hello"
\end{Sinput}
\begin{Soutput}
[1] FALSE
\end{Soutput}
\end{Schunk}
The \code{FALSE} in the last line results from taking into account the
difference between upper-case and lower-case letters.

\index{P}{R Syntax!true@\texttt{TRUE} \& \texttt{FALSE}}
\index{P}{false@\texttt{FALSE}}
\index{P}{true@\texttt{TRUE}}
\index{P}{numerical comparison}


Sometimes you need to combine more than one logical result.  For
example, to ask, ``Is \code{w} between 7 and 12?'' involves combining
two separate questions: ``Is \code{w} greater than 7 AND
is it less than 12?''  In the computer language, this question would
be stated thus:
\begin{Schunk}
\begin{Sinput}
> w > 7 & w < 12
\end{Sinput}
\begin{Soutput}
[1] FALSE FALSE FALSE  TRUE FALSE
\end{Soutput}
\end{Schunk}

There are also logical operators for ``or'' and ``not''.  For
instance, you might ask whether \code{w} is less than 5 OR greater
then 9:
\begin{Schunk}
\begin{Sinput}
> w < 5 | w > 9
\end{Sinput}
\begin{Soutput}
[1]  TRUE  TRUE FALSE  TRUE  TRUE
\end{Soutput}
\end{Schunk}
The ``not'' operator just flips \code{TRUE} and \code{FALSE}:
\begin{Schunk}
\begin{Sinput}
> !(w < 5 | w > 9)
\end{Sinput}
\begin{Soutput}
[1] FALSE FALSE  TRUE FALSE FALSE
\end{Soutput}
\end{Schunk}
As this example illustrates, you can group logical operations in just
the same way as arithmetic operations. 


\index{C}{logical!operators|)}

\subsubsection{Missing Data}

\index{C}{missing data}
\index{C}{data!missing}
\index{C}{NA, missing data}
\index{P}{NA@\texttt{NA}, missing data}
\index{P}{Data!missing (\texttt{NA})}
When recording data from an experiment or an observational study, it
sometimes happens that a particular measurement can't be made or is
lost or is otherwise unavailable.  In R, such \newword{missing data}
can be recorded with the special code \code{NA}.  As you might expect,
arithmetic and other operations on missing data can't be sensibly
performed: giving NA as an input produces NA as an output.
\begin{Schunk}
\begin{Sinput}
> 7 == NA
\end{Sinput}
\begin{Soutput}
[1] NA
\end{Soutput}
\begin{Sinput}
> NA == "Hello"
\end{Sinput}
\begin{Soutput}
[1] NA
\end{Soutput}
\begin{Sinput}
> NA == NA
\end{Sinput}
\begin{Soutput}
[1] NA
\end{Soutput}
\end{Schunk}


In order to test whether there is missing data, a special operator
\code{is.na} can be used:
\begin{Schunk}
\begin{Sinput}
> is.na(NA)
\end{Sinput}
\begin{Soutput}
[1] TRUE
\end{Soutput}
\end{Schunk}
\index{P}{is.na@\texttt{is.na}}
\index{P}{Data!isna@\texttt{is.na}}
\index{P}{missing data!is.na@\texttt{is.na}}

\subsubsection{Collections}

\index{P}{collections!c@\texttt{c}}
\index{P}{c@\texttt{c}}
\index{P}{Data!small collections with \texttt{c}}
R can work with collections of numbers and
character strings.  Some operators work on each item in the
collection, while others combine the items together in some way.  To
illustrate, I'll define three small collections, \code{x}, \code{y},
and \code{fruits}:
\begin{Schunk}
\begin{Sinput}
> x = seq(1, 7)
> x
\end{Sinput}
\begin{Soutput}
[1] 1 2 3 4 5 6 7
\end{Soutput}
\begin{Sinput}
> y = c(7, 8, 9)
> y
\end{Sinput}
\begin{Soutput}
[1] 7 8 9
\end{Soutput}
\begin{Sinput}
> fruits = c("apple", "berry", "cherry")
> fruits
\end{Sinput}
\begin{Soutput}
[1] "apple"  "berry"  "cherry"
\end{Soutput}
\end{Schunk}
The \code{c} operator used in defining \code{y} and \code{fruits} is
useful for creating a small collection ``by hand.''  Often, however,
collections will be created by reading in data from a file or using
some other operator.  I'll introduce these as needed for specific tasks.

\index{C}{arithmetic!on collections}
\index{P}{arithmetic!on collections}
\index{P}{comparison!on collections}

Arithmetic and comparison operators often work item-by-item on the
collection.  For example:
\begin{Schunk}
\begin{Sinput}
> x + 100
\end{Sinput}
\begin{Soutput}
[1] 101 102 103 104 105 106 107
\end{Soutput}
\begin{Sinput}
> sqrt(y)
\end{Sinput}
\begin{Soutput}
[1] 2.65 2.83 3.00
\end{Soutput}
\begin{Sinput}
> fruits == "cherry"
\end{Sinput}
\begin{Soutput}
[1] FALSE FALSE  TRUE
\end{Soutput}
\end{Schunk}
If the operator involves two collections, they have to be the same
size, or R will reuse the smaller collection to match the size of the
larger one:
\begin{Schunk}
\begin{Sinput}
> x == y
\end{Sinput}
\begin{Soutput}
[1] FALSE FALSE FALSE FALSE FALSE FALSE  TRUE
\end{Soutput}
\end{Schunk}
\begin{Schunk}
\begin{Soutput}
Warning message:
In trellis.par.set(theme = col.mosaic()) :
  Note: The default device has been opened to honour attempt to modify trellis settings
\end{Soutput}
\end{Schunk}
The warning message is displayed when some aspect of the computation
is deemed suspect or odd.  Pay attention to such messages since they
may signal that the computation the interpreter carried out is not the
one you intended.  

It's usually obvious what sorts of operators will combine the items
of the collection rather than working on them item by item.  Here are
some examples:
\begin{Schunk}
\begin{Sinput}
> x
\end{Sinput}
\begin{Soutput}
[1] 1 2 3 4 5 6 7
\end{Soutput}
\begin{Sinput}
> y
\end{Sinput}
\begin{Soutput}
[1] 7 8 9
\end{Soutput}
\begin{Sinput}
> mean(x)
\end{Sinput}
\begin{Soutput}
mean 
   4 
\end{Soutput}
\begin{Sinput}
> median(y)
\end{Sinput}
\begin{Soutput}
[1] 8
\end{Soutput}
\begin{Sinput}
> min(x)
\end{Sinput}
\begin{Soutput}
[1] 1
\end{Soutput}
\begin{Sinput}
> max(x)
\end{Sinput}
\begin{Soutput}
[1] 7
\end{Soutput}
\begin{Sinput}
> sum(x)
\end{Sinput}
\begin{Soutput}
[1] 28
\end{Soutput}
\begin{Sinput}
> any(fruits == "cherry")
\end{Sinput}
\begin{Soutput}
[1] TRUE
\end{Soutput}
\begin{Sinput}
> all(fruits == "cherry")
\end{Sinput}
\begin{Soutput}
[1] FALSE
\end{Soutput}
\end{Schunk}

\index{P}{sum@\texttt{sum}}


The \code{length} operator tells how many items there are in the
collection: \index{P}{length}
\begin{Schunk}
\begin{Sinput}
> length(x)
\end{Sinput}
\begin{Soutput}
[1] 7
\end{Soutput}
\begin{Sinput}
> length(y)
\end{Sinput}
\begin{Soutput}
[1] 3
\end{Soutput}
\begin{Sinput}
> length(fruits)
\end{Sinput}
\begin{Soutput}
[1] 3
\end{Soutput}
\end{Schunk}

When there are too many items in a collection to display conveniently
in one line, the R interpreter will break up the display over multiple
lines.  
\begin{Schunk}
\begin{Sinput}
> seq(3, 19)
\end{Sinput}
\begin{Soutput}
 [1]  3  4  5  6  7  8  9 10 11 12 13 14 15 16 17 18 19
\end{Soutput}
\end{Schunk}
At the start of each line, the number in brackets tells the
index of the item that starts that line.  In the above, for instance,
the item 3 is displayed following a \code{[1]} because 3 is the first
item in the collection.  Similarly, the \code{[10]} indicates that the
item 12 is the tenth item in the collection.  The brackets are just
for display purposes; they are not part of the collection itself.

\section{Stuff about Characters}

 Sometimes, you will want to convert
non-text items to text in order to display them in graphs.  There is a
special operator, \code{as.character} for doing this:
\begin{Schunk}
\begin{Sinput}
> as.character(3)
\end{Sinput}
\begin{Soutput}
[1] "3"
\end{Soutput}
\end{Schunk}
The quotes in the output show that it is character type rather then
numeric type.  This isn't terribly important to the human reader, but
the computer regards \code{"3"} as a different thing than \code{3}.
For instance, you can do arithmetic on numbers but not on characters.
\index{P}{conversion!numeric to character}
\index{P}{R Syntax!character text}
\begin{Schunk}
\begin{Sinput}
> 3 + 2
\end{Sinput}
\begin{Soutput}
[1] 5
\end{Soutput}
\end{Schunk}
\begin{Schunk}
\begin{Sinput}
> as.character(3) + 2
\end{Sinput}
\end{Schunk}
\begin{Schunk}
\begin{Soutput}
Error in as.character(3) + 2 : non-numeric argument to binary operator
\end{Soutput}
\end{Schunk}
\index{C}{character string!object type)}
\index{C}{data!character)}




\subsection{Technical Notes: Missing Data}

\index{C}{missing data}

Occasionally, you may have a data frame with missing data.  The
\code{lm} program will handle this sensibly by excluding those cases
where one or more of the variables used in the model are missing.
Those cases with missing data will show up in the lists of residuals 
and fitted values as NAs. (This depends on the
setting of the \code{na.action} argument to \code{lm}.  In the ISM
library, this has been set to \code{na.exclude}.)
\index{P}{missing data}


The NAs can cause a problem in calculations involving the residuals or
the fitted values.  For instance, suppose you have constructed a model
called \code{mymodel} fitted to a data frame with missing data.  Here
is the sum of squares of the fitted values:
\begin{verbatim}
> sum( fitted(mymodel)^2 )
[1] NA
\end{verbatim}
Any NAs in the fitted values propagate through to the overall sum.

\index{P}{na.rm@\texttt{na.rm=}}
\index{P}{Data!na.rm@\texttt{na.rm=}}


There are a couple of ways to deal with this:
\begin{verbatim}
> sum( fitted(mymodel)^2, na.rm=TRUE )
[1] 683434.3
> sum( na.omit(fitted(mymodel))^2 )
[1] 683434.3
\end{verbatim}

\index{P}{na.omit@\texttt{na.omit}}
\index{P}{Data!na.omit@\texttt{na.omit}}

You can also delete any cases with missing data from the data frame
{\em before} fitting the model:
\begin{verbatim}
> cleaned = na.omit(swim)
\end{verbatim}
Be careful, however, since this will exclude any cases that have any
missing variables, even if those variables are not involved in the model.

\index{P}{outlier@\texttt{outlier}*}
\index{P}{Descriptive Statistics!outlier@\texttt{outlier}*}
\index{C}{residual!outliers}
Sometimes it's helpful to look for outliers in the residuals, and to plot
the residuals versus the fitted model values or versus explanatory variables.
For instance:
<<>>=
bwplot( as.numeric(resid(mod1)) )
subset(swim, outlier(resid(mod1)))
xyplot( resid(mod1) ~ fitted(mod1) )
xyplot( resid(mod1) ~ year, data=swim)
@ 
(Technical note: The \code{as.numeric} operator translates the
residuals to a format that \code{bwplot} will work with.)
\index{P}{as.numeric@\texttt{as.numeric}}
\index{P}{Data!as.numeric@\texttt{as.numeric}}

