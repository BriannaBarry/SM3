


COMPARE THIS TO THE GROUP-WISE MODEL Chapter

Emphasize that covariates are important.  There's no reason to ignore
them just because this was common in traditional tests.

Difference between experimental work and field or observational work.

Title of Fisher's book: ``for experimental workers''

\subsubsection{Differences Between Means of Two Groups: the Two-Sample t-test}

The one-way \ANOVA\ approach works when comparing the means among
groups.  It can be applied to any number of groups: 2 groups, 3
groups, 4 groups, and so on.

You might be surprised to hear that there is a special test, called
the \newword{two-sample t-test} for testing whether the means are
different between two groups.  I'll illustrate it using the question
of whether the current population survey data give 
significant evidence that wages are different between men and women
--- two groups.

\index{P}{summary@\texttt{summary}!for linear models}
\index{P}{Hypothesis Testing!summary@\texttt{summary}!for linear models}


Using the modeling approach, the calculations are:
\begin{Schunk}
\begin{Sinput}
> cps = fetchData("cps.csv")
> mod3 = lm( wage ~ sex, data=cps)
> summary(mod3)
\end{Sinput}
\begin{Soutput}
...
            Estimate Std. Error t value Pr(>|t|)    
(Intercept)    7.879      0.322   24.50  < 2e-16 ***
sexM           2.116      0.437    4.84  1.7e-06 ***
---
Signif. codes:  0 '***' 0.001 '**' 0.01 '*' 0.05 '.' 0.1 ' ' 1 

Residual standard error: 5.03 on 532 degrees of freedom
Multiple R-squared: 0.0422,	Adjusted R-squared: 0.0404 
F-statistic: 23.4 on 1 and 532 DF,  p-value: 1.7e-06 
\end{Soutput}
\end{Schunk}
The p-value is very small, justifying rejection of the null
hypothesis.  (This conclusion is subject to some legitimate
criticism, since no attempt has been made to adjust for covariates
such as \VN{age} or the level of education.  With a modeling approach,
you can add these covariates to the model.  But the t-test and one-way
\ANOVA\ approaches don't allow this to be done.  That's sufficient
reason not to use them.  For the purpose here, to demonstrate how to
perform a t-test, I'll continue on without claiming that the use of
the method is justified in anything but a formal sense.)

The streamlined form of the calculation produces the same F and p-value:
\begin{Schunk}
\begin{Sinput}
> anova(mod3)
\end{Sinput}
\begin{Soutput}
Analysis of Variance Table

Response: wage
           Df Sum Sq Mean Sq F value  Pr(>F)    
sex         1    594     594    23.4 1.7e-06 ***
Residuals 532  13483      25                    
---
Signif. codes:  0 '***' 0.001 '**' 0.01 '*' 0.05 '.' 0.1 ' ' 1 
\end{Soutput}
\end{Schunk}

Now to do this in the form of a t-test.  The command uses the familiar
modeling language to identify the response variable (\VN{wage)} and
the explanatory grouping variable (\VN{sex}).
\begin{Schunk}
\begin{Sinput}
> t.test( wage ~ sex, data=cps, var.equal=TRUE)
\end{Sinput}
\begin{Soutput}
	Two Sample t-test

data:  wage by sex 
t = -4.84, df = 532, p-value = 1.703e-06
alternative hypothesis: true difference in means is not equal to 0 
95 percent confidence interval:
 -2.97 -1.26 
sample estimates:
mean in group F mean in group M 
           7.88            9.99 
\end{Soutput}
\end{Schunk}
Rather than computing F, the t-test computes a statistic called the t
value.  The relationship between the two is simple: $t^2 = F$.  So,
the t of $-4.84$ corresponds to F $=23.4$.
Similarly, in the t-test there is a quantity called the ``degrees of
freedom.''  This corresponds to the degrees of freedom in the
denominator in the F test.  The p-values from the t-test and the F
test are the same!

\subsubsection{Technical Note: The Unequal Variance t-test}
\label{sec:unequal-variance-t-test}
\index{C}{t-test!unequal variance}

Actually, the above t-test is not the one that many people would
perform in this situation.  The full name of the above test is the
\newword{equal-variance t-test}. 
I include this section mainly for the
reader who has been told about the 
\newworddef{unequal-variance t-test}{unequal variance t-test}.
The issue of whether to use an equal-variance test or an unequal
variance test is one of the places where the approach taken in this
book rubs against conventional pedagogical practice.  

The term {\bf equal variance} refers to an extension of the null
hypothesis: rather than the null hypothesis being that the population
means of the response variable are the same in the two groups, the
null is expanded to say also that the population variances are equal.
The named argument \code{var.equal=TRUE} is what specifies this
additional assumption to the software.

If you suspend the assumption that the group variances are equal, the
appropriate t-test test takes on another form, the \newword{unequal
  variance t-test}.  Here it is:
\begin{Schunk}
\begin{Sinput}
> t.test( wage ~ sex, data=cps)
\end{Sinput}
\begin{Soutput}
	Welch Two Sample t-test

data:  wage by sex 
t = -4.89, df = 531, p-value = 1.369e-06
alternative hypothesis: true difference in means is not equal to 0 
95 percent confidence interval:
 -2.97 -1.27 
sample estimates:
mean in group F mean in group M 
           7.88            9.99 
\end{Soutput}
\end{Schunk}
The results of the unequal variance t-test are somewhat different from
those of \ANOVA\ or the equal variance t-test. To summarize these 
differences:

\bigskip
\centerline{\begin{tabular}{lccc}
 & & equal var. & unequal var.  \\
 &  F-test & t-test & t-test \\\hline
D.F.& 532 & 532 & 530.5 \\
t & & $-4.84$ & $-4.885$  \\
t$^2$ or F & 23.4 & 23.4 & 23.86 \\
p-value & $1.7\times10^{-6}$  & $1.7\times10^{-6}$ & $1.369\times10^{-6}$\\
\end{tabular}}
\bigskip
The values provided by the equal and unequal variance t-tests 
are different but very close.  

\index{C}{t-test!and adjustment}
\index{C}{adjustment!and t-test}
\index{C}{equal-variance t-test}
\index{C}{unequal variance t-test}

Which test to use?  Many people will argue that the unequal variance
t-test is the appropriate test to use when you don't have a good
reason to assume that the variances of the two groups are different.
This is reasonable enough if one is trying to choose between the
equal-variance and the unequal-variance t-test.  But there is an
``elephant in the room,'' an aspect of the situation that dominates
things but that, for some reason, people don't talk about.  This
elephant is the covariates --- the other explanatory variables 
that you should be adjusting for.  The t-test approach doesn't provide
any capability for such adjustment.  Indeed, if you try to add another
explanatory variable to the t-test command, you get an error message:
\begin{Schunk}
\begin{Sinput}
> t.test( wage ~ sex + educ, data=cps)
\end{Sinput}
\end{Schunk}
\begin{Schunk}
\begin{Soutput}
Error in t.test.formula(wage ~ sex + educ, data = cps) : 
  'formula' missing or incorrect
\end{Soutput}
\end{Schunk}
The same applies to trying to do a t-test with a quantitative
explanatory variable:
\begin{Schunk}
\begin{Sinput}
> t.test( wage ~ educ, data=cps)
\end{Sinput}
\end{Schunk}
\begin{Schunk}
\begin{Soutput}
Error in t.test.formula(wage ~ educ, data = cps) : 
  grouping factor must have exactly 2 levels
\end{Soutput}
\end{Schunk}

\index{C}{ANOVA!covariates}
\index{C}{covariate!ANOVA}

The \ANOVA\ approach easily allows the addition of covariates to the
model.  

If you find yourself in a situation where there are no covariates and
you are comparing two groups with potentially unequal variances, it
makes sense to use the unequal variance t-test.  I regard it as a
special-purpose method that should be used only in such special
situations.  As it happens, the results you get with the
equal-variance t-test and the unequal-variance t-test will be very
close ... unless the variances of the two groups are hugely different.
And, if the variances are indeed so different, you should be asking
yourself deeper questions, such as ``Why am I interested in the group
mean in the first place?  Isn't the difference in variance telling me
something, too?''


\subsubsection{The Grand Mean: the One-Sample t-test}
\index{C}{t-test!one-sample}
\index{C}{t-test!computing}
\index{C}{one-sample t-test}

Occasionally, the question to be asked of a variable is very simple:
Is there evidence that the overall mean is different from some specified
value.  For example, suppose the government had published figures
saying that the mean wage was \$9.25 per hour 
at the time the current population survey
data were collected.  You're interested to know whether the CPS data
are consistent with this claim.  

The one-sample t-test is designed to deal with this situation.
\begin{Schunk}
\begin{Sinput}
> t.test( cps$wage, mu=9.25)
\end{Sinput}
\begin{Soutput}
	One Sample t-test

data:  cps$wage 
t = -1.02, df = 533, p-value = 0.3101
alternative hypothesis: true mean is not equal to 9.25 
95 percent confidence interval:
 8.59 9.46 
sample estimates:
mean of x 
     9.02 
\end{Soutput}
\end{Schunk}
The named argument \code{mu=9.25} specifies the population value of
the mean under the null hypothesis.  In this case, the p-value is
large, $0.3101$, so there is no reason to reject the null hypothesis.

The one-sample t-test is equivalent to the F test applied to a model
where the only explanatory model term is the intercept.  As always,
$t^2 = F$, so the effective F value from the t-test is $-1.016^2 = 1.032$
\begin{Schunk}
\begin{Sinput}
> mod4 = lm( wage - 9.25 ~ 1, data=cps)
> anova(mod4)
\end{Sinput}
\begin{Soutput}
Analysis of Variance Table

Response: wage - 9.25
           Df Sum Sq Mean Sq F value Pr(>F)
Residuals 533  14077    26.4               
\end{Soutput}
\end{Schunk}
> 
The results are identical to the one-sample t-test. (There is no
unequal-variance, one-sample t-test for the simple reason that in the
one-sample t-test there are no groups to have difference variances!)

Notice that in forming the model, 
the null hypothesis parameter, $9.25$ has been subtracted
from the response variable in the modeling statement.  This is
essential.  Otherwise, the test will be whether the sample mean is
consistent with a population value of zero.  Formally, that's a
possible null hypothesis, but it makes no sense in this situation.

One important use for the one-sample t-test occurs in situations where 
measurements have been made in pairs, for example, before-and-after
measurements.  When used this way, the one-sample t-test is called a
\newword{paired t-test}.  The idea is to test the difference within
each pair to see if it is non-zero.  As you'll see in the next
chapter, \ANOVA\ can also be configured to perform such paired tests
and, in fact, is able to generalize them to multiple measurements and
to adjust for covariates.
\index{C}{t-test!paired}


