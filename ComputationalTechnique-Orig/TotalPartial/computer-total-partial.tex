


\subsection{Adjustment}

\index{C}{adjustment!computation}

There are two basic approaches to adjusting for covariates.
Conceptually, the simplest one is to hold the covariates constant at
some level when collecting data or by extracting a subset of data
which holds those covariates constant.  The other approach is to
include the covariates in your models.

For example, suppose you want
to study the differences in the wages of male and females.  The very
simple model \model{\VN{wage}}{\VN{sex}} might give some insight, but
it attributes to \VN{sex} effects that might actually be due to level
of education, age, or the sector of the economy in which the person
works.  Here's the result from the simple model:\datasetCPS
\begin{Schunk}
\begin{Sinput}
> cps = fetchData("cps.csv")
> mod0 = lm( wage ~ sex, data=cps)
> summary(mod0)
\end{Sinput}
\begin{Soutput}
...
            Estimate Std. Error t value Pr(>|t|)    
(Intercept)    7.879      0.322   24.50  < 2e-16 ***
sexM           2.116      0.437    4.84  1.7e-06 ***
---
Signif. codes:  0 ‘***’ 0.001 ‘**’ 0.01 ‘*’ 0.05 ‘.’ 0.1 ‘ ’ 1 

Residual standard error: 5.03 on 532 degrees of freedom
Multiple R-squared: 0.0422,	Adjusted R-squared: 0.0404 
F-statistic: 23.4 on 1 and 532 DF,  p-value: 1.7e-06 
\end{Soutput}
\end{Schunk}
The coefficients indicate that a typical male makes \$2.12 more per
hour than a typical female.  (Notice that $R^2 = 0.0422$ is very small:
\VN{sex} explains hardly any of the person-to-person variability in wage.)

By including the variables \VN{age}, \VN{educ}, and \VN{sector} in the
model, you can adjust for these variables:
\begin{Schunk}
\begin{Sinput}
> mod1 = lm( wage ~ age + sex + educ + sector, data=cps)
> summary(mod1)
\end{Sinput}
\begin{Soutput}
...
              Estimate Std. Error t value Pr(>|t|)    
(Intercept)    -4.6941     1.5378   -3.05  0.00238 ** 
age             0.1022     0.0166    6.17  1.4e-09 ***
sexM            1.9417     0.4228    4.59  5.5e-06 ***
educ            0.6156     0.0944    6.52  1.6e-10 ***
sectorconst     1.4355     1.1312    1.27  0.20500    
sectormanag     3.2711     0.7668    4.27  2.4e-05 ***
sectormanuf     0.8063     0.7311    1.10  0.27064    
sectorother     0.7584     0.7592    1.00  0.31829    
sectorprof      2.2478     0.6698    3.36  0.00085 ***
sectorsales    -0.7671     0.8420   -0.91  0.36273    
sectorservice  -0.5687     0.6660   -0.85  0.39356    
---
Signif. codes:  0 ‘***’ 0.001 ‘**’ 0.01 ‘*’ 0.05 ‘.’ 0.1 ‘ ’ 1 

Residual standard error: 4.33 on 523 degrees of freedom
Multiple R-squared: 0.302,	Adjusted R-squared: 0.289 
F-statistic: 22.6 on 10 and 523 DF,  p-value: <2e-16 
\end{Soutput}
\end{Schunk}
The adjusted difference between the sexes is \$1.94 per hour.  (The
$R^2=0.30$ from this model is considerably larger than for \texttt{mod0},
but still a lot of the person-to-person variation in wages has not
be captured.)

It would be wrong to claim that simply including a covariate in a
model guarantees that an appropriate adjustment has been made.  The
effectiveness of the adjustment depends on whether the model design is
appropriate, for instance whether appropriate interaction terms have
been included.  However, it's certainly the case that if you {\bf don't}
include the covariate in the model, you have {\bf not} adjusted for
it.

The other approach is to subsample the data so that the levels of the
covariates are approximately constant.  For example, here is a subset
that considers workers between the ages of 30 and 35 with between 10
to 12 years of education and working in the sales sector of the
economy:
\begin{Schunk}
\begin{Sinput}
> small = subset(cps, age <=35 & age >= 30 & 
                       educ>=10 & educ <=12 & 
                       sector=="sales" )
\end{Sinput}
\end{Schunk}
The choice of these particular levels of \VN{age}, \VN{educ}, and
\VN{sector} is arbitrary, but you need to choose some level if you
want to hold the covariates appproximately constant.

The subset of the data can be used to fit a simple model:
\begin{Schunk}
\begin{Sinput}
> mod4 = lm( wage ~ sex, data=small)
> summary(mod4)
\end{Sinput}
\begin{Soutput}
...
            Estimate Std. Error t value Pr(>|t|)  
(Intercept)    4.500      0.500     9.0     0.07 .
sexM           4.500      0.866     5.2     0.12  
---
Signif. codes:  0 ‘***’ 0.001 ‘**’ 0.01 ‘*’ 0.05 ‘.’ 0.1 ‘ ’ 1 

Residual standard error: 0.707 on 1 degrees of freedom
Multiple R-squared: 0.964,	Adjusted R-squared: 0.929 
F-statistic:   27 on 1 and 1 DF,  p-value: 0.121 
\end{Soutput}
\end{Schunk}
At first glance, there might seem to be nothing wrong with this
approach and, indeed, for very large data sets it can be effective.
In this case, however, there are only 3 cases that satisfy the various
criteria: two women and one man.
\begin{Schunk}
\begin{Sinput}
> table( small$sex )
\end{Sinput}
\begin{Soutput}
F M 
2 1 
\end{Soutput}
\end{Schunk}

So, the \$4.50 difference between
the sexes and wages depends entirely on the data from a single male!
(Chapter \ref{chap:confidence} describes how to assess the precision
of model coefficients.  This one works out to be $4.50 \pm 11.00$ ---
not at all precise.)
