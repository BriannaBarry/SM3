

 
Regression reports are generated using software you have already encountered: \code{lm} to fit a model and \code{summary} to construct the report from
the fitted model.  To illustrate: \datasetSwimming
\begin{Schunk}
\begin{Sinput}
> swim = fetchData("swim100m.csv")
> mod = lm(time ~ year + sex, data=swim)
> summary(mod)
\end{Sinput}
\begin{Soutput}
...
            Estimate Std. Error t value Pr(>|t|)    
(Intercept) 555.7168    33.7999   16.44  < 2e-16 ***
year         -0.2515     0.0173  -14.52  < 2e-16 ***
sexM         -9.7980     1.0129   -9.67  8.8e-14 ***
---
Signif. codes:  0 '***' 0.001 '**' 0.01 '*' 0.05 '.' 0.1 ' ' 1 

Residual standard error: 3.98 on 59 degrees of freedom
Multiple R-squared: 0.844,	Adjusted R-squared: 0.839 
F-statistic:  160 on 2 and 59 DF,  p-value: <2e-16 
\end{Soutput}
\end{Schunk}


\subsection{Confidence Intervals from Standard Errors}

\index{P}{Confidence Intervals}

Given the coefficient estimate and the standard error
from the regression report, the confidence interval is easily generated.  
For a 95\% confidence interval, you just multiply the standard error by 2 
to get the margin of error.  For example, in the above, the
margin of error on \indicatorVar{sex}{M} is $2 \times 1.013 = 2.03$, or, in 
computer notation:
\begin{Schunk}
\begin{Sinput}
> 2 * 1.0129
\end{Sinput}
\begin{Soutput}
[1] 2.03
\end{Soutput}
\end{Schunk}

If you want the two endpoints of the confidence interval, rather than 
just the margin of error, do these
simple calculations: (1) subtract the margin of error from the estimate; 
(2) add the margin of error to the estimate.  So, 
\begin{Schunk}
\begin{Sinput}
> -9.798 - 2*1.0129
\end{Sinput}
\begin{Soutput}
[1] -11.8
\end{Soutput}
\begin{Sinput}
> -9.798 + 2*1.0129
\end{Sinput}
\begin{Soutput}
[1] -7.77
\end{Soutput}
\end{Schunk}
The key thing is to remember the multiplier that is applied 
to the standard error.  A multiplier of approximately 2 is for
a 95\% confidence level.  

The \function{confint} function provides a convenient way to calculate
confidence intervals directly.
\index{C}{confidence interval!computation}
\index{P}{confint@\texttt{confint}}
\index{P}{Confidence Intervals!confint@\texttt{confint}}
It calculates the exact multiplier (which depends somewhat on the
sample size)  and applies it to the standard error to produce the confidence intervals.
\begin{Schunk}
\begin{Sinput}
> mod = lm(time ~ year + sex, data=swim)
> confint(mod)
\end{Sinput}
\begin{Soutput}
              2.5 %  97.5 %
(Intercept) 488.083 623.350
year         -0.286  -0.217
sexM        -11.825  -7.771
\end{Soutput}
\end{Schunk}

It would be convenient if the regression report included confidence
intervals rather than the standard error.  Part of the reason it
doesn't is historical: the desire to connect to the traditional by-hand calculations.



\subsection{Bootstrapping Confidence Intervals}

Confidence intervals on model coefficients can be computed using the
same bootstrapping technique introduced in Chapter \ref{chap:statistical-inference}.

Start with your fitted model. To illustrate, here is a model of
swimming time over the years, taking into account sex:
\begin{Schunk}
\begin{Sinput}
> swim = fetchData("swim100m.csv")
> lm(time ~ year + sex, data=swim)
\end{Sinput}
\begin{Soutput}
...
(Intercept)         year         sexM  
    555.717       -0.251       -9.798  
\end{Soutput}
\end{Schunk}
These coefficients reflect one hypothetical draw from the
population-based sampling
distribution.  It's impossible to get another draw from the
``population'' here: the actual records are all you've got.

But to approximate sampling variation, you can treat the sample as
your population and re-sample:
\begin{Schunk}
\begin{Sinput}
> lm(time ~ year + sex, data=resample(swim))
\end{Sinput}
\begin{Soutput}
...
(Intercept)         year         sexM  
    495.600       -0.221       -8.717  
\end{Soutput}
\end{Schunk}

Constructing many such re-sampling trials and collect the results 
\begin{Schunk}
\begin{Sinput}
> s = do(500) * lm(time ~ year + sex, data=resample(swim))
\end{Sinput}
\end{Schunk}
\begin{Schunk}
\begin{Sinput}
> s
\end{Sinput}
\end{Schunk}
\begin{Schunk}
\begin{Soutput}
  Intercept   year   sexM sigma r-squared
1       531 -0.239  -9.60  3.71     0.853
2       514 -0.230 -10.25  3.28     0.893
3       525 -0.236  -9.70  3.67     0.827
4       562 -0.255  -9.41  3.97     0.846
5       568 -0.257 -10.76  4.65     0.819
... for 500 cases altogether ...
\end{Soutput}
\end{Schunk}

To find the standard error of the coefficients, just take the standard
deviation across the re-sampling trials:
\begin{Schunk}
\begin{Sinput}
> sd(s)
\end{Sinput}
\begin{Soutput}
Intercept      year      sexM     sigma r-squared 
  43.5426    0.0220    0.9530    0.8057    0.0343 
\end{Soutput}
\end{Schunk}

Multiplying the standard error by 2 gives the approximate 95\% margin
of error.  Alternatively, you can use the \function{confint} function
to calculate this for you:
\begin{Schunk}
\begin{Sinput}
> confint(s, method="stderr")
\end{Sinput}
\begin{Soutput}
       name    2.5%   97.5%
1 Intercept 472.355 643.454
2      year  -0.296  -0.209
3      sexM -11.621  -7.876
4     sigma   2.215   5.381
5 r-squared   0.788   0.923
\end{Soutput}
\end{Schunk}



\subsection{Prediction Confidence Intervals}

\index{C}{confidence interval!for prediction}
\index{C}{prediction!confidence interval}

When a model is used to make a prediction, it's helpful to be able to
describe how precise the prediction is.  For instance, suppose you \datasetFeet
want to use the \texttt{kidsfeet.csv} data set to make a prediction of
the foot width of a girl whose foot length is 25 cm. 

\index{P}{predict@\texttt{predict}}
\index{P}{Modeling!predict@\texttt{predict}}

First, fit your model:
\begin{Schunk}
\begin{Sinput}
> feet = fetchData("kidsfeet.csv")
> names(kids)
\end{Sinput}
\begin{Soutput}
 [1] "Gender"      "Grade"       "Age"         "Race"       
 [5] "Urban.Rural" "School"      "Goals"       "Grades"     
 [9] "Sports"      "Looks"       "Money"      
\end{Soutput}
\begin{Sinput}
> levels(kids$sex)
\end{Sinput}
\begin{Soutput}
NULL
\end{Soutput}
\begin{Sinput}
> mod = lm(width ~ length + sex, data=feet)
\end{Sinput}
\end{Schunk}

Now apply the model to the new data for which you want to make a
prediction.  Take care to use the right coding for categorical variables.
\begin{Schunk}
\begin{Sinput}
> predict(mod, newdata=data.frame( length=25, sex="G" ))
\end{Sinput}
\begin{Soutput}
   1 
8.93 
\end{Soutput}
\end{Schunk}

In order to generate a confidence interval, the \code{predict}
operator needs to be told what type of interval is wanted.  There are
two types of prediction confidence intervals:
\begin{description}
\item[Interval on the model value] which reflects the sampling
  distributions of the coefficients themselves.  To calculate this,
  use the \code{interval="confidence"} named argument:
\begin{Schunk}
\begin{Sinput}
> predict(mod, newdata=data.frame( length=25, sex="G" ), 
         interval="confidence")
\end{Sinput}
\begin{Soutput}
   fit  lwr  upr
1 8.93 8.74 9.13
\end{Soutput}
\end{Schunk}

The components named \code{lwr} and \code{upr} are the lower and upper
limits of the confidence interval, respectively.
\index{P}{interval@\texttt{interval=}!in \texttt{predict}}
\index{P}{Confidence Intervals!interval@\texttt{interval=}!in \texttt{predict}}

\item[Interval on the prediction] which includes the variation due to
  the uncertainty in the coefficients as well as the size of a typical
  residual.  To find this interval, use the
  \code{interval="prediction"} named argument:
\begin{Schunk}
\begin{Sinput}
> predict(mod, newdata=data.frame( length=25, sex="G" ), 
         interval="confidence")
\end{Sinput}
\begin{Soutput}
   fit  lwr  upr
1 8.93 8.74 9.13
\end{Soutput}
\end{Schunk}

The prediction interval is larger than the model-value confidence
interval because the residual always gives additional uncertainty
around the model value.
\end{description} 


