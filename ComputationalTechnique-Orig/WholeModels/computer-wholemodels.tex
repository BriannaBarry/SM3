



The chapter presents two ways of doing hypothesis tests on whole
models: (1) permutation tests where the connection is severed between
the explanatory and response variables, (2) tests such as \ANOVA\
where the sampling distribution is calculated from first principles.
In practice, first-principle tests are used most of the time.
Still, the permutation test is useful for developing intuition about
hypothesis testing --- our main purpose here --- 
and for those not-so-rare occasions where the assumptions
behind the first-principle tests are dubious.

\subsection{The Permutation Test}

The idea of a permutation test is to enforce the null hypothesis that
there is no connection between the response variables and the
explanatory variables.  An effective way to do this is to randomize
the response variable in a way that is consistent with sampling
variability.  When constructing confidence intervals, the
\function{resampling} function was used.  Re-sampling will typically
repeat some cases and omit others.  Here, the \function{shuffle}
function will be used instead, to scramble the order of one or more
variables while leaving the others in their original state.


To illustrate, consider a model for exploring whether \VN{sex} and
\VN{mother}'s height are related to the height of the child: \datasetGalton
\begin{Schunk}
\begin{Sinput}
> galton = fetchData("galton.csv")
> mod = lm(height ~ sex + mother, data=galton)
> coefficients(mod)
\end{Sinput}
\begin{Soutput}
(Intercept)        sexM      mother 
     41.450       5.177       0.353 
\end{Soutput}
\end{Schunk}
The coefficients indicate that typical males are taller than typical 
females by about 5 inches and that for each inch taller the mother is,
a child will typically be taller by 0.35 inches.  
A reasonable test statistic to summarize the whole model is $R^2$:
\begin{Schunk}
\begin{Sinput}
> rsquared(mod)
\end{Sinput}
\begin{Soutput}
[1] 0.562
\end{Soutput}
\end{Schunk}

For confidence intervals, re-sampling was applied to the entire data
frame.  This selects random cases, but each selected case is an
authentic one that matches exactly the original values for that case.
The point of re-sampling is to get an idea of the variability
introduced by random sampling of authentic cases.
\begin{Schunk}
\begin{Sinput}
> do(5) * lm(height ~ sex + mother, data=resample(galton))
\end{Sinput}
\begin{Soutput}
  Intercept sexM mother sigma r-squared
1      42.3 5.37  0.337  2.29     0.593
2      42.1 4.93  0.344  2.34     0.550
3      43.5 5.23  0.321  2.36     0.568
4      37.9 5.26  0.409  2.34     0.579
5      39.9 5.14  0.378  2.29     0.570
\end{Soutput}
\end{Schunk}
The \indicatorVar{sex}{M} coefficients are tightly grouped near 5
inches, the \VN{mother} coefficients are around 0.3 to 0.4.

\index{P}{resample@\texttt{resample}*}
\index{P}{shuffle@\texttt{shuffle}*}
\index{P}{Hypothesis Testing!resample@\texttt{resample}*}
\index{P}{Hypothesis Testing!shuffle@\texttt{shuffle}*}

\index{P}{Resampling!resample@\texttt{resample}*}
\index{P}{Resampling!shuffle@\texttt{shuffle}*}
\index{C}{permutation test!computing}

In order to carry out a permutation test, do not randomize the
whole data frame.  Instead, shuffle just the response variable:
\begin{Schunk}
\begin{Sinput}
> do(5) * lm(shuffle(height) ~ sex + mother, data=galton)
\end{Sinput}
\begin{Soutput}
  Intercept     sexM  mother sigma r-squared
1      63.7 -0.30663  0.0507  3.58   0.00300
2      72.3  0.00676 -0.0859  3.58   0.00306
3      70.8  0.00563 -0.0629  3.58   0.00164
4      70.2  0.02282 -0.0533  3.58   0.00119
5      68.3 -0.55719 -0.0192  3.58   0.00613
\end{Soutput}
\end{Schunk}
Now the \indicatorVar{sex}{M} and \VN{mother} 
coefficients are close to zero, as would be expected when
there is no relationship between the response variable and the
explanatory variables.

In constructing the sampling distribution under the null hypothesis,
you should do hundreds of trials of fitting the model to the scrambled
data, calculating the test statistic ($R^2$ here) for each trial.  Note that each
trial itself involves all of the cases in your sample, but those cases
have been changed so that the shuffled variable almost certainly
takes on a different value in every case than in the original data.
\begin{Schunk}
\begin{Sinput}
> nulltrials = do(500) *  rsquared(
     lm(shuffle(height) ~ sex + mother, data=galton))
\end{Sinput}
\end{Schunk}
Notice that the \function{rsquared} operator has been used to calculate
the test statistic $R^2$ from the model.  The output of \function{do}
is a data frame: 
\begin{Schunk}
\begin{Sinput}
> nulltrials
\end{Sinput}
\end{Schunk}
\begin{Schunk}
\begin{Soutput}
       var
1 0.000920
2 0.001854
3 0.000533
... for 500 cases altogether ...
\end{Soutput}
\end{Schunk}
Naturally, all of the $R^2$ values for the trials are close to zero.
After all,  there is no
relation between the response variable (after randomization with
\function{shuffle}) and the explanatory variables.

The p-value can be calculated directly from the trials, by comparison
to the observed value in the actual data: $R^2$ was 0.5618. 
\begin{Schunk}
\begin{Sinput}
> count(var > 0.5618, nulltrials)
\end{Sinput}
\begin{Soutput}
count.FALSE 
        500 
\end{Soutput}
\end{Schunk}
None of the 500 trials were greater than the value of the test statistic, 0.5618.
It wouldn't be fair to claim that $p=0$, since we only did 500 trials,
but it is reasonable to say that the permutation test shows the 
p-value is $p \leq 1/500$.


\subsection{First-Principle Tests}

\index{C}{regression report!hypothesis testing}
\index{C}{hypothesis testing!in regression report}
On modern computers, the permutation test is entirely practical.  But
a few decades ago, it was not.  Great creativity was applied to
finding test statistics where the sampling distribution could be
estimated without extensive calculation.  One of these is the F
statistic.  This is still very useful today and is a standard part of
the regression report in many statistical packages.  

\index{P}{summary@\texttt{summary}!for linear models}
\index{P}{Hypothesis Testing!summary@\texttt{summary}!for linear models}

Here is the regression report from the
\model{\VN{height}}{\VN{sex}+\VN{mother}}:
\begin{Schunk}
\begin{Sinput}
> mod = lm( height ~ sex + mother, data=galton)
> summary(mod)
\end{Sinput}
\begin{Soutput}
...
            Estimate Std. Error t value Pr(>|t|)    
(Intercept)  41.4495     2.2095    18.8   <2e-16 ***
sexM          5.1767     0.1587    32.6   <2e-16 ***
mother        0.3531     0.0344    10.3   <2e-16 ***
---
Signif. codes:  0 '***' 0.001 '**' 0.01 '*' 0.05 '.' 0.1 ' ' 1 

Residual standard error: 2.37 on 895 degrees of freedom
Multiple R-squared: 0.562,	Adjusted R-squared: 0.561 
F-statistic:  574 on 2 and 895 DF,  p-value: <2e-16 
\end{Soutput}
\end{Schunk}
The last line of the report
shows an F statistic of 574 based on an $R^2$ of 0.562 and
translates this to a p-value that is practically zero: \code{<2e-16}.

By way of showing that the regression report is rooted in the same
approach shown in the chapter, you can confirm the calculations.
There are $m=3$ coefficients and $n=898$ cases,
producing $n-m=895$ degrees of freedom in the denominator and $m-1=2$
degrees of freedom in the numerator. The calculation of the F
statistic from $R^2$ and the degrees of freedom follows the formula
given in the chapter.  
$$ F = \frac{R^2}{m-1} \Biggm/ \frac{1-R^2}{n-m} $$
Plugging the values into the formula
\begin{Schunk}
\begin{Sinput}
> (0.562/2) / ((1-.562)/895)
\end{Sinput}
\begin{Soutput}
[1] 574
\end{Soutput}
\end{Schunk}
F is the test statistic.  To convert it to a p-value, you need to
calculate how extreme the value of F$=574.2$ is with reference to the
F distribution with 895 and 2 degrees of freedom.
\begin{Schunk}
\begin{Sinput}
> 1 - pf( 574.2, 2, 895)
\end{Sinput}
\begin{Soutput}
[1] 0
\end{Soutput}
\end{Schunk}
\index{P}{pf@\texttt{pf}}
\index{P}{Hypothesis Testing!pf@\texttt{pf}}
\index{C}{p value!from F statistic}
The calculation of p-values from F always follows this form.  In the
context of the F distribution, ``extreme'' always means ``bigger
than.''  So, calculate the area under the F distribution to the right
of the observed value.

