Density plots are particularly useful for comparing distributions.
For example, here is a graph that displays the distribution of both
the adult children's heights and their mother's heights.  This is done
with two commands, one to draw the density plot for the mothers'
heights and the second for the children's heights.  
\begin{verbatim}
> plot( density(galton$mother), 
     main="Children & Mothers", 
     xlab="Height (inches)",
     xlim=c(55,75), 
     col="red")
> lines( density( galton$height), col='blue')
\end{verbatim}
\noindent \graphicsfile[width=2.5in]{density2.pdf}
This example illustrates how the construction of a graphic can become
complicated.  The \code{plot} operator was used in the first command
in order to start a new graphic, erasing any previous graphics.  
The second command used \code{lines}
in order to add on to the existing graphic.  R set the scales for the
axes automatically based on the range of the data used in the
\code{plot} command, but these turned out to be inappropriate for the
data in the second command.  So, the \code{xlim} argument was used to
set the scale of the $x$ axis to run from 55 to 75.

The labels for the curves were added this way:
\begin{verbatim}
> text(locator(1), "Children")
> text(locator(1), "Mothers")
\end{verbatim}
The \code{locator(1)} statement causes R to activate the mouse so that
a position can be marked on the graph.  This position is then passed
as an argument to \code{text} to add the text string to the graphic at
that point.
