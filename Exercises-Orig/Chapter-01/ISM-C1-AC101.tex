%% Author: Daniel Kaplan
%% Subject: R (basic commands)

NOTE: Remember to load the \texttt{mosaic} package before starting:
\begin{verbatim}
> load(mosaic)
\end{verbatim}

\bigskip

Each of the following statements has a syntax mistake.  Write the
statements properly and give a sentence saying what was wrong. (Cut
and paste the correct statement from R, along with any output that R
gives and your sentence saying what was wrong in the original.)

\noindent Here's an example: 

\begin{quotation}
\noindent QUESTION: What wrong with this statement?
\begin{verbatim}
> a = fetchData(myfile.csv) 
\end{verbatim}

\noindent ANSWER: {\em It should be 
\begin{verbatim}
> a = fetchData("myfile.csv") 
\end{verbatim}
The file name is a character string and
therefore should be in quotes.  Otherwise it's treated as an object
name, and there is no object called myfile.csv.}
\end{quotation}

Now for the real thing.  Say what's wrong with each of these
statements for the purpose given:

\begin{enumerate}[(a)]
\item \verb|> seq(5;8)|  to give \verb|[1] 5 6 7 8|
%\begin{MultipleChoice}
%\wrong{Nothing is wrong.}
%\correct{Use a comma instead of a semi-colon to separate arguments: \texttt{seq(5,8)}.}
%\wrong{It should be \texttt{seq(5 to 8)}.}
%\end{MultipleChoice}
\TextEntry

\begin{AnswerText}
The semi-colon is not a valid separator between arguments.  The statement should be \code{seq(5,8)}.
\end{AnswerText}

\item \verb|> cos 1.5| to calculate the cosine of 1.5 radians

\TextEntry

\begin{AnswerText}
The parentheses have been left out.  It should be \code{cos(1.5)}
\end{AnswerText}


\item \verb|> 3 + 5 = x| to  make x take the value 3+5

\TextEntry

\begin{AnswerText}
Assignment statements should have the object name on the left: \code{x=3+5}.
\end{AnswerText}


\item \verb|> sqrt[4*98]| to find the square root of 392

\TextEntry

\begin{AnswerText}
Use parentheses, not square brackets, to invoke an operator: \code{sqrt(4*98)}.

\end{AnswerText}


\item \verb|> 10,000 + 4,000| adding two numbers to get 14,000

\TextEntry

\begin{AnswerText}
Don't punctuate numbers with commas, just the digits: \code{10000+4000}.
\end{AnswerText}


\item \verb|> sqrt(c(4,16,25,36))=4| intended to give 
\begin{verbatim}[1] FALSE TRUE FALSE FALSE
\end{verbatim}

\TextEntry

\begin{AnswerText}
Only a single \code{=} was used, the assignment operator rather than equality.  It should be \code{sqrt(c(4,16,25,36)) == 4}
\end{AnswerText}


\item \verb|> fruit = c(apple, berry, cherry)| to create
a collection of names 
\begin{verbatim}
[1] "apple" "berry" "cherry"
\end{verbatim}

\TextEntry

\begin{AnswerText}
The character strings have to be quoted to distinguish them from object names.  It should be \code{fruit = c( "apple", "berry", "cherry")}.
\end{AnswerText}


\item \verb|> x = 4(3+2)| where x is intended to be assigned the value 20

\TextEntry

\begin{AnswerText}
The multiplication operator needs to be specified explicitly and not implicitly as in algebraic notation.  The statement should be \code{x = 4*(3+2)}.
\end{AnswerText}



\item \verb|> x/4 = 3+2| where x is intended to be assigned the value 20

\TextEntry

\begin{AnswerText}
Only an object name can be on the left side of the assignment operator.  It should be \code{x = 4*(3+2)}.
\end{AnswerText}


\end{enumerate}

