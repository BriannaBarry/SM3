%% Author: Daniel Kaplan
%% Subject: Computation (sequences)

According to legend, when the famous mathematician Johann Carl Friedrich Gauss
(1777-1855) was a child, one of his
teachers tried to distract him with busy work: 
add up the numbers 1 to 100.   Gauss did this easily and
immediately without a computer.  But using the computer, which of the
following commands will do the job?

\begin{MultipleChoice}
\wrong{sum(1,100)}
\wrong{seq(1,100)}
\wrong{sum of seq(1,100)}
\correct{sum(seq(1,100))}
\wrong{seq(sum(1,100))}
\wrong{sum(1,seq(100))}
\end{MultipleChoice}

\begin{AnswerText}
A just sums two numbers: $1$ and $100$; B only gives the sequence from $1$ to $100$ without summing them; C was written in human language, not computer (R) language; for E, the inside summation is performed first (giving a value of $101$), and so the entire command produces a sequence from $1$ to $101$; for F, a sequence from $1$ to $100$ is produced first, then it sums every element of this sequence AND $1$, with $1$ being redundant. Therefore, only D produces the correct output.
\end{AnswerText}
