%\begin{AcroScoreProblem}{read-01}
%\renewcommand{\acroBackReference}{Intro~Reading}

\input{../ISM-macros}

\begin{enumerate}


\item How can a model be useful even if it is not exactly correct?\\
\TextEntry[itemname=useful]

\item Give an example of a model used for classification.\\
\TextEntry[itemname=classif]

\item Often we describe personalities as ``patient,'' ``kind,'' 
  ``vengeful,'' etc.  How can these descriptions be used as models for
  prediction? \\
\TextEntry[itemname=personal]

\item Give three examples of models that you use in everyday life.
  For each, say what is
  the purpose of the model and in what ways the representation
  differs from the real thing. \\
\TextEntry[itemname=everyday]



\item Make a sensible statement about how precisely these quantities are
  typically measured:
\begin{itemize}
\item The speed of a car.
\item Your weight.
\item The national unemployment rate.
\item A person's intelligence.
\end{itemize}
\TextEntry[itemname=precision]


\begin{AnswerText}
A car speedometer is typically precise to about 1 mile per hour for a
car going 50 mph,
weights to 1 to 2 pounds for a 100 pound person , the unemployment rate to 1\% (although the
rate is quoted to tenths of percentage points).  It's hard to find
detailed information about intelligence tests.   The SAT, a college
admissions test scored from 0 to 2400, apparently has a
reproducibility of about plus or minus 100 points.

\end{AnswerText}


\item Give an example of a controlled experiment.  What quantity or quantities
  have been varied and what has been held
  constant?\\
\TextEntry[itemname=experiment]



\item Using one of your textbooks from another field, pick an
  illustration or diagram. Briefly describe the illustration and 
  explain how this is a model, in what ways
  it is faithful to the system being described and in what ways it
  fails to reflect that system. \TextEntry[itemname=textbook]
\end{enumerate}
%\end{AcroScoreProblem}