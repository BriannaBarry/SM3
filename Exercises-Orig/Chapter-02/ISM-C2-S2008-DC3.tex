%% Author: Daniel Kaplan
%% Subject: Types of variables (categorical, ordinal, quantitative)

Sometimes categorical information is represented numerically.  In the
early days of computing, it was very common to represent everything
with a number.  For instance the categorical variable for sex, with
levels male or female, might be stored as 0 or 1.  Even categorical
variables like \variableName{race} or \variableName{language}, with
many different levels, can be represented as a number.  The codebook
provides the interpretation of each number (hence the word
``codebook'').

Here is a very small part of a dataset from the 1960s used to study the influence of smoking 
and other factors on the weights of babies at birth.\cite{nolan-speed-2001} \datasetGestation

\bigskip
\centerline{\begin{tabular}{rrrrrrrr}
gest. & wt & race & ed & wt.1 & inc & smoke & number\\
284 & 120 & 8 & 5 & 100 & 1 & 0 & 0\\
282 & 113 & 0 & 5 & 135 & 4 & 0 & 0\\
279 & 128 & 0 & 2 & 115 & 2 & 1 & 1\\
244 & 138 & 7 & 2 & 178 & 98 & 0 & 0\\
245 & 132 & 7 & 1 & 140 & 2 & 0 & 0\\
351 & 140 & 0 & 5 & 120 & 99 & 3 & 2\\
282 & 144 & 0 & 2 & 124 & 2 & 1 & 1\\
279 & 141 & 0 & 1 & 128 & 2 & 1 & 1\\
281 & 110 & 8 & 5 & 99 & 2 & 1 & 2\\
273 & 114 & 7 & 2 & 154 & 1 & 0 & 0\\
285 & 115 & 7 & 2 & 130 & 1 & 0 & 0\\
255 & 92 & 4 & 7 & 125 & 1 & 1 & 5\\
261 & 115 & 3 & 2 & 125 & 4 & 1 & 5\\
261 & 144 & 0 & 2 & 170 & 7 & 0 & 0\\
\end{tabular}}

\bigskip

At first glance, all of the data seems quantitative.  But read the codebook:

\begin{verbatim}
gest. - length of gestation in days

wt -  birth weight in ounces (999 unknown)

race - mother's race 
   0-5=white 6=mex 7=black 8=asian 
   9=mixed 99=unknown 

ed - mother's education 
   0= less than 8th grade, 
   1 = 8th -12th grade - did not graduate, 
   2= HS graduate--no other schooling , 
   3= HS+trade,
   4=HS+some college 
   5= College graduate, 
   6&7 Trade school HS unclear, 
   9=unknown 

marital 1=married, 2= legally separated, 3= divorced,
  4=widowed, 5=never married

inc - family yearly income in $2500 increments 
  0 = under 2500, 1=2500-4999, ..., 
  8= 12,500-14,999, 9=15000+,
  98=unknown, 99=not asked

smoke - does mother smoke? 0=never, 1= smokes now, 
    2=until current pregnancy, 3=once did, not now, 
    9=unknown

number  - number of cigarettes smoked per day 
   0=never, 1=1-4, 2=5-9, 3=10-14, 4=15-19, 
   5=20-29, 6=30-39, 7=40-60, 
   8=60+, 9=smoke but don't know, 98=unknown, 99=not asked
\end{verbatim}


Taking into account the codebook, what kind of data is each variable?
If the data have a natural order, but are not genuinely quantitative,
say ``ordinal.''  You can ignore the ``unknown'' or ``not asked''
codes when giving your answer.

\begin{enumerate}[(a)]
\item Gestation  \SelectSetHoriz[itemname=gest]{quantitative}{categorical,ordinal,quantitative}
\item Race  \SelectSetHoriz[itemname=race]{categorical}{categorical,ordinal,quantitative}
\item Marital  \SelectSetHoriz[itemname=marital]{categorical}{categorical,ordinal,quantitative}
\item Inc  \SelectSetHoriz[itemname=income]{ordinal}{categorical,ordinal,quantitative}
\item Smoke   \SelectSetHoriz[itemname=smoke]{categorical}{categorical,ordinal,quantitative}
\item Number   \SelectSetHoriz[itemname=number]{ordinal}{categorical,ordinal,quantitative}
\end{enumerate}

\begin{AnswerText}
Gestation, given in days, is an ordinary quantitative variable.  Race has no natural ordering and so is categorical.  Similarly for marital status.  Income has been broken into categories, but these have a natural order.  Smoke has no natural order, so it's categorical.  Number is divided up into categories, but these have a natural order.
\end{AnswerText}

The disadvantage of storing categorical information as numbers is that
it's easy to get confused and mistake one level for another.  Modern
software makes it easy to use text strings to label the different
levels of categorical variables.  Still, you are likely to encounter
data with categorical data stored numerically, so be alert.

A good modern practice is to code missing data in a consistent way that can be automatically recognized by 
software as meaning missing.  Often, NA is used for this purpose.  Notice that in the \VN{number} variable, there is a 
clear order to the categories until one gets to level 9, which means ``smoke but don't know.''  This is an 
unfortunate choice.  It would be better to store \VN{number} as a quantitative variable telling the number of cigarettes smoked per day.  Another variable could be used to indicate whether missing data was ``smoke but don't know,'' ``unknown'', or ``not asked.''  
