%% Author: Daniel Kaplan
%% Subject: Coverage intervals (computing)


Which one of these commands will give the 95th percentile of the
children's heights in Galton's data? \datasetGalton

\begin{MultipleChoice}
\wrong{\code{quantile(galton$height, 95)}} 
\correct{\code{quantile(galton$height, 0.95)}} 
\wrong{\code{quantile(galton$father, 95)}} 
\wrong{\code{quantile(galton$father, 0.95)}} 
\end{MultipleChoice}

\begin{AnswerText}
What's key here is to remember that percentiles need to be specified
as a fraction: the 95th percentile is 0.95.

The children's heights are contained in the \VN{height} variable, not
\VN{father} (which has the heights of the fathers).

To make the command work, you will have to read in the Galton data
that is contained in the ISM software distribution.  (See the
instructions in the book for getting this file and loading it into R.)
You can store the data in an object called ``galton'' so that your
commands will match those given here, but of course you could use any
name you like.
\begin{verbatim}
> galton = ISMdata("galton.csv")
\end{verbatim}
\end{AnswerText}

\bigskip

\noindent Which of these command will give the 90-percent coverage interval of the children's heights in Galton's data?

\begin{MultipleChoice}
\correct{\code{quantile( galton$height, c(0.05, 0.95))}}
\wrong{\code{quantile( galton$height, c(0.025, 0.975))}}
\wrong{\code{quantile( galton$height, 0.90)}}
\wrong{\code{quantile( galton$height, 90)}}
\end{MultipleChoice}

\begin{AnswerText}
Since you're looking for a 90-percent coverage interval to cover the
center of the data, you want to
leave 5 percent on the left and 5 percent on the right.  This means
that the interval will run from the 5th percentile to the 95th percentile.
\end{AnswerText}

\bigskip
\noindent 
Find the 50-percent coverage interval of the following variables in Galton's height data:
\begin{itemize}
\item Father's heights
\begin{MultipleChoice}
\wrong{59 to 73 inches}
\correct{68 to 71 inches}
\wrong{63 to 65.5 inches}
\wrong{68 to 74 inches}
\end{MultipleChoice}

\begin{AnswerText}
A 50-percent coverage interval will leave 25 percent of the data to
the left and another 25 percent to the right, so the interval runs
from the 25th to the 75th percentile.
\begin{verbatim}
> quantile( galton$father, c(0.25, 0.75))
25% 75% 
 68  71 
\end{verbatim}
Note how the command uses \code{c(0.25, 0.75)} to create a set with
BOTH percentiles.  If you wanted, you could do one at a time:
\begin{verbatim}
> quantile( galton$father, 0.25 )
25% 
 68 
> quantile( galton$father, 0.75 )
75% 
 71 
\end{verbatim}
But it's nicer to have both ends of the interval reported in one statement.
\end{AnswerText}

\item Mother's heights
\begin{MultipleChoice}
\wrong{59 to 73 inches}
\wrong{68 to 71 inches}
\correct{63 to 65.5 inches}
\wrong{68 to 74 inches}
\end{MultipleChoice}

\begin{AnswerText}
Just like with the fathers' heights, but now ask for the quantiles on
the mothers' height:
\begin{verbatim}
> quantile( galton$mother, c(0.25, 0.75))
 25%  75% 
63.0 65.5
\end{verbatim}
\end{AnswerText}

\end{itemize}


Find the 95-percent coverage interval of
\begin{itemize}
\item Father's heights
\begin{MultipleChoice}
\wrong{65 to 73 inches}
\correct{65 to 74 inches}
\wrong{68 to 73 inches}
\wrong{59 to 69 inches}
\end{MultipleChoice}

\begin{AnswerText}
For a 95-percent coverage interval, look at the 2.5 and 97.5
percentile.
\begin{verbatim}
> quantile( galton$father, c(0.025, 0.975))
 2.5% 97.5% 
   65    74 
\end{verbatim}
\end{AnswerText}

\item Mother's heights
\begin{MultipleChoice}
\wrong{62.5 to 68.5 inches}
\correct{65 to 69 inches}
\wrong{63 to 68.5 inches}
\correct{59 to 69 inches}
\end{MultipleChoice}

\begin{AnswerText}
\begin{verbatim}
> quantile( galton$mother, c(0.025, 0.975))
 2.5% 97.5% 
   59    69 
\end{verbatim}
\end{AnswerText}
\end{itemize}

