%% Author: Daniel Kaplan
%% Subject: Computation (data manipulation)




\providecommand{\HCode}[1]{#1} % dummy, just in case it's PDFlatex
\HCode{<link rel="stylesheet" href="fixSweave.css" type="text/css"
  media="screen" />}
\HCode{<link rel="stylesheet" href="http://dl.dropbox.com/u/5098197/Math135/RGuide/fixSweave.css" type="text/css"
 media="screen" />}


Here are some useful operators for taking a quick look at data frames:

\bigskip
\centerline{\begin{tabular}{lp{3in}}
\code{names} & Lists the names of the components.\\
\code{ncol} & Tells how many components there are.\\
\code{nrow} & Tells how many lines of data there are.\\
\code{head} & Prints the first several lines of the data frame.\\
\end{tabular}}
\bigskip

\noindent Here are some examples of these commands applied to the \code{CO2}
data frame:
\begin{Schunk}
\begin{Sinput}
  CO2 = fetchData("CO2")
\end{Sinput}
\begin{Soutput}
Called from: fetchData("CO2")
\end{Soutput}
\begin{Sinput}
  names(CO2)
\end{Sinput}
\begin{Soutput}
[1] "Plant"     "Type"      "Treatment" "conc"      "uptake"   
\end{Soutput}
\begin{Sinput}
  ncol(CO2)
\end{Sinput}
\begin{Soutput}
[1] 5
\end{Soutput}
\begin{Sinput}
  nrow(CO2)
\end{Sinput}
\begin{Soutput}
[1] 84
\end{Soutput}
\begin{Sinput}
  head(CO2)
\end{Sinput}
\begin{Soutput}
  Plant   Type  Treatment conc uptake
1   Qn1 Quebec nonchilled   95   16.0
2   Qn1 Quebec nonchilled  175   30.4
3   Qn1 Quebec nonchilled  250   34.8
4   Qn1 Quebec nonchilled  350   37.2
5   Qn1 Quebec nonchilled  500   35.3
6   Qn1 Quebec nonchilled  675   39.2
\end{Soutput}
\end{Schunk}

\begin{itemize}

\item The data frame \code{iris} records measurements on flowers. You can read in with 
\begin{Schunk}
\begin{Sinput}
  iris = fetchData("iris")
\end{Sinput}
\begin{Soutput}
Called from: fetchData("iris")
\end{Soutput}
\end{Schunk}
creating an object named \code{iris}.

Use the above operators to answer the following questions.

\begin{enumerate}
\item Which of the following is the name of a column in \code{iris}?\\

\SelectSetHoriz{Species}{flower,Color,Species,Length}

\item How many rows are there in \code{iris}?\\
\SelectSetHoriz{150}{1,50,100,150,200}

\item How many columns are there in \code{iris}?\\
\SelectSetHoriz{5}{2,3,4,5,6,7,8,10}

\item What is the \code{Sepal.Length} in the third row?\\
\SelectSetHoriz{4.7}{1.2,3.6,4.2,4.7,5.9}

\end{enumerate}

\item The data frame \code{mtcars} has data on cars from the 1970s.  
You can read it in with 
\begin{Schunk}
\begin{Sinput}
  cars = fetchData("mtcars")
\end{Sinput}
\begin{Soutput}
Called from: fetchData("mtcars")
\end{Soutput}
\end{Schunk}
creating an object named \code{cars}.

Use the above operators to answer the following questions.
\begin{enumerate}
\item Which of the following is the name of a column in \code{cars}?

\SelectSetHoriz{carb}{carb,color,size,weight,wheels}

\item How many rows are there in \code{cars}?\\
\SelectSetHoriz{32}{30,31,32,33,34,35}

\item How many columns are there in \code{cars}?\\
\SelectSetHoriz{11}{7,8,9,10,11}

\item What is the \code{wt} in the second row?\\
\SelectSetHoriz{2.875}{2.125,2.225,2.620,2.875,3.215}
\end{enumerate}
\end{itemize}


