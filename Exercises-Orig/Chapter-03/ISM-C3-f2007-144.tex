%% Author: Daniel Kaplan
%% Subject: descriptive stats (estimating center and spread)

Here are some familiar quantities.  For each of them, indicate what is
a typical value, how far a typical case is from this typical value,
and what is an extreme but not impossible case.

Example: Adult height.  Typical value, 1.7 meters (68 inches).  Typical case
is about 7cm (3 inches) from the typical value.  An extreme height is
2.2 meters (87 inches).

\begin{itemize}

\item An adult's weight.\TextEntry[itemname=weight]

\begin{AnswerText}
Let's answer for women.  A typical weight is about 130 lbs, with a
typical spread about 20 lbs from the center.  An extreme weight is 350
lbs on the high side and 80 lbs on the low side.
\end{AnswerText}

\item Income of a full-time employed person.\TextEntry[itemname=income]

\begin{AnswerText}
Writing this in 2011, a typical income for a full-time employed person
is around \$40,000 USD per year.  Typical spread is about \$20,000 USD
from the center.  Given minimum wage laws, an extreme case on the low
side is about \$15,000 USD.  On the high side, the sky is the limit,
with some people having an income of billions of dollars per year.
\end{AnswerText}

\item Speed of cars on a highway in good conditions.\TextEntry[itemname=speed]

\begin{AnswerText}
A typical car goes about 60 mph in the US (depending on the state and
the highway).  Cars typically go at roughly the same speed, so the
typical spread might be just 3-5 mph above or below the center value.
Speeders rarely go above 100 mph --- that's an extreme value.

If there is a traffic jam, the situation is completely different.  The
typical speed might be 10 mph, with 0 not being particularly extreme.
\end{AnswerText}

\item Systolic blood pressure in adults.  [You might need to look this
  up on the Internet.]\TextEntry[itemname=bp]

\begin{AnswerText}
A typical systolic BP is roughly 110 mmHg, which a typical spread of
perhaps 10-20 mmHG on either side.  An extremely high blood pressure
is 200 mmHg.  A systolic blood pressure below 90 mmHg is considered
abnormally low and can lead to dizziness or fainting.
\end{AnswerText}

\item Blood cholesterol LDL levels. [Again, you might need the
  Internet.]\TextEntry[itemname=cholesterol]

\begin{AnswerText}
Web sites list an LDL cholesterol level of less than 200 mg/dL as
``desirable,'' without indicating what fraction of the population is
in this group and what a typical value is.  They list a level of 240
mg/dL as ``high risk'', again without saying what fraction of people
are at high risk.  
\end{AnswerText}

\item Fuel economy among different models of
  cars.\TextEntry[itemname=car]

\begin{AnswerText}
A typical car gets something like 20 mile-per-gallon, with most cars
getting somewhere from 10 to 35.  An extreme low value for a car is 5
mpg.  A very high value is 50 mpg.
\end{AnswerText}

\item Wind speed on a summer day.\TextEntry[itemname=wind]

\begin{AnswerText}
A typical summer day has a wind speed of 0 to 15 miles per hour
(depending of course on where you are).  So the typical value might be
5 mph with a spread of about 5 around that.  0 is not so unusual, and
of course you can't have a negative wind speed.  A speed of 60 mph is
quite extreme.  Above that is hurricane levels, even more extreme.  
\end{AnswerText}

\item Hours of sleep per night for college students.\TextEntry[itemname=sleep]

\begin{AnswerText}
Of course, some college students go without sleep for a day or two,
but it seems reasonable to think that most students get an average of
5 to 10 hours per night.  Perhaps 6-7 is typical (although not
necessarily healthy!), with a spread of 2 hours on either side of
that.  An extreme value is 15 hours --- though this can happen.
Obviously 24 is the limit.
\end{AnswerText}

\end{itemize}

