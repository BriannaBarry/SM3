%% Author: DTK,
%% Subject: outliers and transformation

The identification of a case as an outlier does not always mean that
the case is invalid or abnormal or the result of a mistake.  One
situation where perfectly normal cases can look like outliers is when
there is a mechanism of proportionality at work.  Imagine, for
instance, that there is a typical rate of production of a substance,
and the normal variability is proportional in nature, say from 1/10 of
that typical rate to 10 times the rate.  This leads to a situation
where some normal cases are 100 times as large as others.

To illustrate, look at the \code{alder.csv} data set, which contains
field data from a study of nitrogen fixation in alder plants.  The
\VN{SNF} variable records the amount of nitrogen fixed in soil by
bacteria that reside in root nodules of the plants.  Make a box plot
and a histogram and describe the distribution.  Which of the following
descriptions is most appropriate:

\begin{MultipleChoice}
\wrong{The distribution is skewed to the left, with outliers at very
low values of \VN{SNF}.}
\correct{The distribution is skewed to the right, with outliers at very
high values of \VN{SNF}.}
\wrong{The distribution is roughly symmetrical, although there are a
few outliers.}
\end{MultipleChoice}

In working with a variable like this, it can help to convert the
variable in a way that respects the idea of a proportional change. 
For instance, consider the three numbers 0.1, 1.0, and 10.0, which are
evenly spaced in proportionate terms --- each number is 10 times
bigger than the preceding number.  But as absolute differences, 0.1
and 1.0 are much closer to each other than 1.0 and 10.0.

The {\em logarithm} function transforms numbers to a scale where even
proportions are equally spaced.  For instance, taking the logarithm of the
numbers 0.1, 1.0, and 10.0 gives the sequence $-1$, 0, 1 --- exactly
evenly spaced.  

The \VN{logSNF} variable gives the logarithm of \VN{SNF}.  Plot out
the distribution of \VN{logSNF}.  Which of the following descriptions
is most apt?
\begin{MultipleChoice}
\wrong{The distribution is skewed to the left.}
\wrong{The distribution is skewed to the right.}
\correct{The distribution is roughly symmetrical.}
\end{MultipleChoice}

You can compute logarithms directly in R, using the functions
\code{log}, \code{log2}, or \code{log10}.  Which of these functions
was used to compute the quantity \VN{logSNF} from \VN{SNF}. (Hint: Try
them out!)

\SelectSetHoriz{log}{log, log2, log10}

The {\em base} of the logarithm gives the size of the proportional
change that corresponds to a 1-unit increase on the logarithmic
scale.  For example, \code{log2} calculates the base-2 logarithm.  On
the base-2 logarithmic scale, a doubling in size corresponds to a
1-unit increase.  In contrast, on the base-10 scale, a ten-fold
increase in size gives a 1-unit increase. 

Logarithmic transformations are often used to deal with variables that
are positive and strongly skewed.  In economics, price, income and production
variables are often this way.  In general, any variable where it is
sensible to describe changes in terms of proportion might be better
displayed on a logarithmic scale.  For example, price inflation rates
are usually given as percent (e.g., ``The inflation rate was 4\% last
year.'') and so in dealing with prices over time, the logarithmic
transformation can be appropriate.
