%% Author: Daniel Kaplan
%% Subject: Variability and typical

An advertisement for ``America's premier weight loss destination''
states that ``a typical two week stay results in a loss of 7-14 lbs.''
(The {\em New Yorker}, 7 April 2008, p 38.)  

The advertisement gives no details about the meaning of ``typical.''
Give two or three plausible interpretations of the quoted 7-14 pound
figure in terms of ``typical.''  What interpretation would be most
useful to a person trying to predict how much weight he or she might
lose?

\TextEntry

% See s2008-conf6 for another take on this problem

\begin{AnswerText}
A standard meaning for ``typical'' is the center of the distribution,
as might be measured by the mean or median.  Judging from the ad, the
mean or median might be 10.5 pounds, but stating this would
misleadingly imply too high a level of precision.  The statement
``7-14 lbs'' indicates that the weight loss varies significantly from person to
person.  If ``typical'' is taken to refer to the varation in weight
loss, the interval might be pointing to how much a typical person
varies from the mean, as could be measured by the standard deviation.  

An obvious interpretation is that a ``typical'' person's weight
loss will fall into the range 7-14 pounds.  It seems reasonable that
this interval should cover the majority of people involved.  At a
minimum, it should therefore be a 50\% coverage interval.

The interval would be most useful to a prospective client if it covers
a substantial majority of the people involved, say 75\% or 95\%.

\end{AnswerText}